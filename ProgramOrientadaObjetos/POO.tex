\documentclass[11pt,a4paper]{article}
\usepackage[utf8]{inputenc}
\usepackage{amsmath}
\usepackage{amsfonts}
\usepackage{amssymb}
\usepackage{graphicx}
\usepackage[left=2cm,right=2cm,top=2cm,bottom=2cm]{geometry}
\usepackage{musixtex}
\title{Programacion Orientada a Objetos}
\author{Lara}
\date{\today}
\begin{document}
	\maketitle
	\section{Objetos y clases}
	un programa informático en un lenguaje orientado a objetos, estaremos creando en
	nuestro equipo un modelo de una cierta parte del mundo. Los componentes a partir de los cuales
	se construye el modelo son los objetos que aparecen en el dominio del problema concreto que
	analicemosEsos objetos deben representarse en el modelo informático que estemos desarrollando. Los objetos pueden clasificarse, y una clase sirve para describir, de una manera abstracta, todos los
	objetos de un tipo concreto. nos referimos a cada objeto particular con el nombre de instancia. hablaremos de instancias cuando queramos insistir en el hecho
	de que se trata de objetos de una clase determinada
	El área de la parte
	inferior de la pantalla en la que se muestra el objeto se denomina banco de objetos (object bench).\\
	\\
	el nombre de un método y los tipos de parámetros presentes en su cabecera
	reciben el nombre de signatura del método. lasoperaciones que podemos emplear para manipular el círculo se denominan métodos. los métodos se llaman o invocan. los valores adicionalesrequeridos por algunos métodos se denominan parámetros.\\ \textbf{void movehorizontal(int distance)} eso se denomina cabezera del método, la pare enre parénesis es el parámetro requerido definido por un nombre y un tipo \\
	La signatura que acabamos
	de mostrar indica que el método requiere un parámetro de tipo int denominado distance.
	El nombre proporciona una pista acerca del significado de los datos que se espera que se introduzcan.
	En conjunto, el nombre de un método y los tipos de parámetros presentes en su cabecera
	reciben el nombre de signatura del método.
	\\
	\\
	\textbf{Tipos de datos: }Un tipo especifica cuáles son los datos que pueden traspasarse a un parámetro. El tipo int hace
	referencia a los números enteros. Si un método no tiene ningún
	parámetro, el nombre del método irá seguido por un par de paréntesis vacíos. Si dispone un parámetro, mostrará el tipo y el nombre de dicho parámetro.
	Los comentarios se incluyen para proporcionar
	información para el lector (humano) del programa. \\
	Una vez que disponemos de una clase, podemos crear tantos objetos (o instancias) de dicha
	clase como queramos. Podemos cambiar un atributo
	de un objeto (por ejemplo, su tamaño) invocando un método sobre dicho objeto. Esto afectará a
	dicho objeto concreto, pero no a los restantes objetos de la misma clase.\\
	\\
	\textbf{estado:} El conjunto de valores de todos los atributos que definen un objeto se denomina también estado del
	objeto.\\
	En BlueJ, el estado de un objeto puede inspeccionarse seleccionando la función Inspect en el menú
	emergente del objeto. Cuando se inspecciona un objeto, se muestra lo que se denomina un inspector
	de objetos (object inspector). El inspector de objetos es una vista ampliada del objeto, en la que
	se muestran los atributos almacenados dentro del mismo\\
	Algunos métodos, al ser invocados, modifican el estado de un objeto. Java denomina campos a esos atributos de los objetos. todos los objetos de la misma clase tienen los
	mismos campos. Es decir, el número, el tipo y los nombres de los campos son idénticos, mientras
	que los valores concretos de cada campo particular de cada objeto pueden ser diferentes. Por el
	contrario, los objetos de clases diferentes pueden tener diferentes campos. La razón es que el número, los tipos y los nombres de los campos se definen dentro de una clase,
	no en un objeto. Lo mismo cabe decir de los métodos. Los métodos se definen en la clase del objeto. Como resultado,
	todos los objetos de una misma clase tendrán los mismos métodos. Sin embargo, los métodos
	se invocan sobre objetos concretos.\\
	\textbf{código java}\\
	Cuando programamos en Java, lo que hacemos esencialmente es escribir instrucciones para
	invocar métodos sobre los objetos
	observaciones:\\
	Vemos en qué consiste el proceso de creación y de denominación de un objeto. Técnicamente,
	lo que estamos haciendo es almacenar el objeto Person en una variable; hablaremos de esto en
	detalle en el siguiente capítulo.\\
	Apreciamos que, para llamar a un método de un objeto, lo que hacemos es escribir el nombre
	del objeto seguido de un punto y seguido del nombre del método. El comando termina con una
	lista de parámetros, o con un par de paréntesis vacíos si no hay parámetros.\\
	Todas las instrucciones Java terminan con un punto y coma.\\ \\
	\textbf{interacción entre objetos}
	si queremos realizar una secuencia de tareas en Java, normalmente no lo haremos
	a mano. En lugar de ello, crearemos una clase que lo haga por nosotros.\\
	Lo importante aquí es que los objetos pueden crear otros objetos y pueden invocar también los
	métodos de otros objetos. En un programa Java normal puede haber perfectamente cientos o miles
	de objetos. El usuario del programa se limita a iniciar el programa (lo que normalmente hace que
	se cree un primer objeto) y todos los demás objetos son creados, directa o indirectamente, por
	dicho objeto. Cada clase tiene un cierto código fuente asociado. El código fuente es el texto que define los
	detalles de la clase. El código fuente es texto escrito en el lenguaje de programación Java. Define los campos y métodos
	que tiene una clase y define también qué es exactamente lo que sucede cuando se invoca cada
	método. \\
	los métodos
	pueden devolver un valor como resultado. De hecho, la cabecera de cada método nos dice si el
	método devuelve o no un resultado y cuál es el tipo de ese resultado. Los métodos con valores de retorno nos permiten obtener información de un objeto mediante la
	invocación de un método. Esto significa que podemos emplear métodos para cambiar el estado
	de un objeto o para averiguar cuál es ese estado.\\
	los objetos pueden pasarse como parámetros a los métodos de otros
	objetos. Cuando un método espera un objeto como parámetro, la signatura del método especifica
	como tipo de parámetro el nombre de la clase del objeto esperado.\\
	\\
	
	\textbf{resumen:}\\
	En este capítulo, hemos explorado los fundamentos de las clases y de los objetos. Hemos explicado
	el hecho de que los objetos se especifican mediante clases. Las clases representan el concepto
	general de las cosas, mientras que los objetos representan instancias concretas de una clase. Podemos
	tener múltiples objetos de cualquier clase determinada.\\
	\\
	Los objetos disponen de métodos que utilizamos para comunicarnos con ellos. Podemos emplear
	un método para efectuar un cambio en el objeto o para obtener información del objeto. Los métodos
	pueden tener parámetros y los parámetros tienen sus correspondientes tipos. Los métodos
	tienen tipos de retorno, que especifican el tipo de dato que van a devolver. Si el tipo de retorno es
	void, entonces es que no devuelven nada.\\
	\\
	Los objetos almacenan los datos en campos (que también tienen tipos). El conjunto de todos los
	valores de datos de un objeto se conoce como estado del objeto.\\
	\\
	Los objetos se crean a partir de definiciones de clases que han sido escritas en un lenguaje de programación
	concreto. Buena parte de la tarea de programación en Java está relacionada con cómo
	escribir esas definiciones de clases. Un programa Java de gran tamaño tendrá muchas clases, cada
	una de las cuales contará con varios métodos, que pueden llamarse unos a otros de varias formas
	distintas.\\
	\\
	Para desarrollar programas Java, necesitamos aprender a escribir definiciones de clases, entre ellas
	sus campos y métodos, y a ensamblar estas clases correctamente. El resto de este libro se ocupa
	precisamente de estas cuestiones.
	Términos introducidos en el capítulo:
	objeto, clase, instancia, método, signatura, parámetro, tipo, estado, código fuente,
	valor de retorno, compilador
	\section{definicion de clases}
	los elementos básicos de las definiciones de clase: campos, constructores, parámetros y
	métodos. Los métodos contienen instrucciones.\\
	\\
	El texto de una clase puede dividirse en dos partes principales: un envoltorio exterior que simplemente
	da nombre a la clase y una parte interna, mucho más larga, que se encarga de realizar todo
	el trabajo. El envoltorio exterior de las diferentes clases se parece bastante. Ese envoltorio exterior contiene
	la cabecera de la clase, cuyo propósito principal es proporcionar a la clase un nombre. De acuerdo
	con un convenio ampliamente aceptado, los nombres de las clases comienzan siempre con una
	letra mayúscula. Siempre que se emplee de manera constante, este convenio permite distinguir
	fácilmente los nombres de las clases de otros tipos de nombres, como los nombres de variables y
	los nombres de métodos, que describiremos más adelante. Encima de la cabecera de la clase se
	incluye un comentario (en texto de color azul) que indica alguna observación sobre la clase.\\
	\\
	\textbf{palabras clave:}Las palabras “public” y “class” forman parte del lenguaje Java 
	A términos como “public” y
	“class” los denominamos palabras clave o palabras reservadas; ambos términos se utilizan con
	frecuencia y de manera intercambiable. En Java existen unas 50 palabras de este tipo, y el lector
	pronto se acostumbrará a reconocer la mayor parte de ellas. Un aspecto que conviene recordar
	es que las palabras clave Java nunca contienen letras mayúsculas, mientras que las que elegimos
	como programadores (por ejemplo, “TicketMachine”) son a menudo una mezcla de mayúsculas y
	minúsculas.
	\\
	\\
	\textbf{parte interna de la clase: campos, consrucores  méodos}
	\\
	En la parte interna de la clase definimos los campos, los constructores y los métodos que proporcionan
	a los objetos de dicha clase sus características y comportamientos propios. Resumimos las
	características esenciales de estos tres componentes de una clase de la forma siguiente:
	\begin{itemize}
		\item Los campos almacenan datos de manera persistente dentro de un objeto.
		\item 
		Los constructores son responsables de garantizar que un objeto se configure apropiadamente en
		el momento de crearlo por primera vez.
		\item 
		Los métodos implementan el comportamiento de un objeto; proporcionan su funcionalidad. 
	\end{itemize}
	Los campos también se conocen con el
	nombre de variables de instancia, porque la palabra variable se utiliza como término general
	para todos aquellos elementos que permiten almacenar datos en un programa. Los campos son pequeñas cantidades de espacio dentro de un objeto que pueden emplearse para
	almacenar datos de manera persistente. Todos los objetos tendrán espacio para cada campo declarado
	en su clase. Cada campo dispone de su propia declaración en el código fuente. Dentro de la definición completa
	de la clase. los campos siempre se definen como privados (private). Como los campos pueden almacenar valores que varíen con el tiempo, se conocen también con
	el nombre de variables. En caso necesario, el valor almacenado en un campo puede modificarse
	con respecto a su valor inicial. \\
	cada vez que definamos una variable de campo dentro
	de una clase: \textbf{private + tipo(int, String...) + nombre + ;}\\
	\\
	\textbf{Constructores:} Los constructores tienen un papel especial que cumplir. Son responsables de garantizar que cada
	objeto se configure adecuadamente en el momento de crearlo por vez primera. En otras palabras,
	garantizan que cada objeto esté listo para ser utilizado inmediatamente después de su creación.
	Este proceso de construcción también se denomina inicialización.
	Una de las características distintivas de los constructores es que tienen el mismo nombre de la
	clase en la se encuentran definidos. En general, el nombre del constructor
	sigue inmediatamente a la palabra public, sin ningún otro elemento entre ellos. Cabe esperar que exista una estrecha conexión entre lo que sucede en el cuerpo de un constructor
	y en los campos de la clase. Esto se debe a que uno de los papeles principales del constructor es
	el de inicializar los campos. o pasado como parametros el valor del campo necesario para el nuevo objeto o inicializando los campos a una constante siempre que se cree un objeto de esa clase.\\
	Los constructores y los métodos desempeñan papeles muy distintos en la vida de un objeto, pero
	la forma en que ambos reciben valores desde el exterior es la misma: a través de parámetros. Los parámetros son otro tipo de variable, igual que los campos, por lo que se utilizan para almacenar
	datos. Los parámetros son variables que se definen en la cabecera de un constructor o de un
	método. Los parámetros se emplean como
	una especie de mensajeros temporales, que transportan datos cuyo origen se sitúa fuera del constructor
	o método y que hacen que esos datos estén disponibles en el interior del constructor o
	método.\\
	espacio adicional para el objeto que solo se crea cuando el constructor se ejecuta.
	Lo denominaremos espacio del constructor del objeto (o espacio del método cuando hablemos
	acerca de métodos y no de constructores), ya que en aquel caso la situación es exactamente la
	misma. El espacio del constructor se utiliza para proporcionar espacio en el que almacenar los
	valores de los parámetros del constructor. En nuestros diagramas, todos las variables se representan
	mediante recuadros blancos. Distinguiremos entre los nombres de los parámetros dentro de un constructor o método y los valores
	externos de los parámetros; denominaremos a los nombres parámetros formales y a los valores
	parámetros reales. Un parámetro formal solo está disponible para un objeto dentro del cuerpo de un constructor o
	método que lo declare. Decimos que el ámbito de un parámetro está restringido al cuerpo del
	constructor o método en el que se declara. Por el contrario, el ámbito de un campo es todo el conjunto
	de la definición de la clase: puede accederse a él desde cualquier punto de la misma clase.
	Se trata de una diferencia muy importante entre estos dos tipos de variables.\\
	Un concepto relacionado con el ámbito de las variables es el tiempo de vida de las mismas.
	El tiempo de vida de un parámetro está limitado a una única llamada a un constructor o método.
	Cuando se invoca un constructor o método, se crea el espacio adicional para las variables de parámetro
	y los valores externos se copian en dicho espacio. Una vez que la llamada ha completado su
	tarea, los parámetros formales desaparecen y los valores que contenían se pierden. En otras palabras,
	cuando el constructor ha terminado de ejecutarse, se elimina todo el espacio del constructor),
	junto con las variables de parámetro contenidas dentro del mismo.\\
	Por el contrario, el tiempo de vida de un campo coincide con el del objeto al que pertenece.
	Cuando se crea un objeto, se crean también todos los campos del mismo, y esos campos persisten
	mientras dure el tiempo de vida del objeto.\\
	Al igual que cabía esperar que existiera una estrecha conexión entre un constructor y los campos
	de una clase, también es de esperar que exista una estrecha conexión entre los parámetros del
	constructor y los campos, porque a menudo se necesitarán valores externos para configurar los
	valores iniciales de uno o más de esos campos. Cuando suceda así, los tipos de los parámetros se
	asemejarán estrechamente a los tipos de los campos correspondientes.\\
	\\
	Una de las cosas que puede que haya observado es que los nombres de variables que utilizamos
	para los campos y los parámetros tienen una estrecha conexión con el propósito de la variable.\\
	\\
	Las instrucciones de asignación funcionan
	tomando el valor que aparece en el lado derecho del operador y copiando dicho valor en la variable
	especificada en el lado izquierdo. El lado
	derecho se denomina expresión. En su forma más general, las expresiones son elementos que calculan
	un valor, pero en este caso la expresión consiste en una sola variable, cuyo valor se copia
	en la variable. Una regla relativa a las instrucciones de asignación es que el tipo de la expresión del lado derecho
	debe corresponderse con el tipo de la variable a la que se asigna.\\Esta misma regla también se aplica entre parámetros formales y reales:
	el tipo de una expresión de parámetro real debe corresponderse con el tipo de la variable que actúa
	como parámetro formal.\\
	\\
	\textbf{Métodos} Los métodos tienen dos partes: una cabecera y un cuerpo. Es importante distinguir entre las cabeceras de los métodos y las declaraciones de los campos, porque
	pueden parecer bastante similares. las cabeceras de los métodos siempre incluyen una pareja de paréntesis –“(” y “)”– y no incluyen
	punto y coma al final de la cabecera. El cuerpo del método es el resto del método, es decir, el código situado después de la cabecera.
	Siempre se encierra entre un par de llaves Los cuerpos de los métodos contienen las
	declaraciones y las instrucciones que definen lo que hace un objeto cuando se invoca ese método.
	Las declaraciones se utilizan para crear espacio adicional de variables temporales, mientras que las
	instrucciones describen las acciones del método\\
	Cualquier conjunto de declaraciones e instrucciones situado entre una pareja de llaves se conoce
	con el nombre de bloque.\\
	El método tiene un tipo de retorno mientras que el constructor no tiene ningún tipo de
	retorno. El tipo de retorno se escribe justo delante del nombre del método. Esta es una diferencia
	que se aplica en todos los casos. En Java, una regla que se aplica de manera general es que los constructores no pueden tener un
	tipo de retorno. Por otro lado, tanto los constructores como los métodos pueden tener cualquier
	número de parámetros formales, incluyendo ninguno.\\
	Cuando un método contiene una instrucción de retorno, será siempre la instrucción final de dicho
	método, porque una vez que se ejecute dicha instrucción de retorno no se podrá ejecutar ninguna
	instrucción adicional en ese método.
	Los tipos de retorno y las instrucciones de retorno funcionan conjuntamente.\\
	\\
	\textbf{Métodos selectores y mutadores}:  se denominan métodos selectores (o simplemente selectores) devuelven al llamante información acerca del estado de un objeto; proporcionan acceso a información
	sobre el estado del objeto. Un selector suele contener una instrucción de retorno, para poder
	devolver dicha información.\\
	Existe confusión acerca de lo que realmente significa “devolver un valor”. A menudo se tiende
	a creer que significa que el programa imprime algo, pero no es así , En realidad, devolver
	un valor significa que se transfiere una cierta información internamente entre dos partes diferentes
	del programa. Una parte del programa ha solicitado la información de un objeto mediante la invocación de un método y el valor de retorno es la forma que tiene el objeto de devolver dicha
	información al llamante.\\
	\\
	De la misma forma que podemos pensar en una llamada a un selector como si fuera una solicitud de
	información (una pregunta), veremos una llamada a un mutador como si fuera una solicitud para
	que un objeto cambie su estado. La forma más básica de mutador es aquella que admite un único
	parámetro cuyo valor se utiliza para sobrescribir directamente lo que haya almacenado en uno de
	los campos del objeto. Como complemento directo de los métodos “get”, este conjunto de métodos
	se denominan a menudo métodos “set”. Un efecto distintivo de un mutador es que un objeto mostrará a menudo un comportamiento ligeramente
	distinto antes y después de invocar a ese mutador.\\
	Un tipo de retorno void indica que el método no devuelve ningún valor al llamante.
	Este tipo de retorno es significativamente distinto a todos los demás tipos de retorno. En BlueJ,
	la diferencia más destacable es que no se muestra ningún cuadro de diálogo de valor de retorno
	después de una llamada a un método void. Dentro del cuerpo de un método void, esta diferencia
	se refleja en el hecho de que no hay instrucción de retorno\\
	\\
	\textbf{impresion desde métodos}: una
	instrucción como: System.out.println("algo");\\ imprime literalmente la cadena de caracteres que aparece entre la pareja de caracteres de dobles
	comillas. al método println
	del objeto System.out que está incorporado en el lenguaje Java, y lo que aparece entre los paréntesis
	es el parámetro de cada llamada al método, como cabría esperar. En println usamos nombres literales, llamados literal de cadena con compillas y si queremos imprimir el valor de por ejemplo un nombre del campo, este no lo ponemo on comillas pq queremos su valor, no su nombre.\\
	Cuando se utiliza entre una cadena y cualquier otra cosa, “+” es un operador de concatenación de
	cadenas (es decir, concatena o junta cadenas de caracteres con el fin de crear una nueva cadena)
	y no un operador de suma aritmética.\\
	\\
	\textbf{Resumen sobre métodos:} Los métodos implementan las acciones fundamentales realizadas por los objetos. Un método con parámetros recibirá los datos que se le transfieran desde la entidad que invoca a
	ese método y usará dichos datos para poder llevar a cabo una tarea concreta. Sin embargo,
	no todos los métodos utilizan parámetros; muchos hacen uso simplemente de los datos almacenados
	en los campos del objeto para llevar a cabo su tarea.\\
	Si un método tiene un tipo de retorno distinto de void, devolverá algún dato al lugar desde el que
	fue invocado, y dicho dato será utilizado, casi con total seguridad, en el llamante para realizar
	cálculos adicionales o para controlar la ejecución del programa. Muchos métodos, sin embargo,
	tienen un tipo de retorno void y no devuelven nada, aunque realizan una tarea útil dentro del contexto
	de su objeto.
	Los métodos\\ selectores tienen tipos de retorno distintos de void y devuelven información acerca
	del estado de un objeto. Los métodos mutadores modifican el estado de un objeto. Los mutadores
	suelen tener parámetros, cuyos valores se utilizan en la modificación, aunque es perfectamente
	posible escribir un método mutador que no admita ningún parámetro.\\
	Hemos visto que la clase tiene
	una pequeña capa externa que proporciona un nombre a la clase y un cuerpo interno de mayor
	tamaño que contiene campos, un constructor y varios métodos. Los campos se utilizan para almacenar
	datos que permiten a los objetos mantener un estado que persiste entre llamadas sucesivas
	a los métodos. Los constructores se utilizan para configurar un estado inicial cuando se crea el
	objeto. Disponer de un estado inicial apropiado permitirá a los objetos responder adecuadamente
	a las llamadas a métodos que se produzcan inmediatamente después de la creación de esos objetos.
	Los métodos implementan el comportamiento definido para los objetos pertenecientes a esa
	clase. Los métodos selectores proporcionan información acerca del estado de un objeto y los métodos
	mutadores modifican el estado de un objeto.\\
	Hemos visto que los constructores se distinguen de los métodos porque tienen el mismo nombre
	que la clase en la que están definidos. Tanto los constructores como los métodos pueden aceptar
	parámetros, pero solo los segundos pueden tener un tipo de retorno. Los tipos de retorno distintos
	de void nos permiten pasar un valor desde el interior de un método hacia el lugar desde el que el
	método fue invocado. Un método con un tipo de retorno distinto de void debe tener al menos una
	instrucción de retorno dentro de su cuerpo; a menudo, dicha instrucción será la última del método.
	Los constructores nunca tienen un tipo de retorno, ni siquiera void.\\
	\\
	\textbf{insrucciones condicionales:} Las instrucciones condicionales también se conocen con el nombre de instrucciones if, debido
	a la palabra clave usada en la mayoría de los lenguajes de programación para implementarlas.
	Una instrucción condicional nos permite llevar a cabo una de dos acciones posibles con base en
	el resultado de una prueba o comprobación. Si la comprobación es verdadera entonces hacemos
	una cosa; en caso contrario, hacemos algo distinto. \\
	\\
	BlueJ muestra el código fuente con algunos detalles de formato adicionales:
	concretamente, sitúa recuadros coloreados alrededor de algunos elementos\\
	\\
	Estas indicaciones de color se conocen con el nombre de representación visual del ámbito y pueden
	ayudarnos a clarificar las unidades lógicas del programa. Un ámbito (también denominado
	bloque) es una unidad de código que normalmente está encerrada entre llaves. El cuerpo completo
	de una clase es un ámbito, como también lo son el cuerpo de cada método y las partes if y else de
	una instrucción condicional.
	Como puede ver, los ámbitos están a menudo anidados: la instrucción if se encuentra dentro de
	un método, que a su vez se encuentra dentro de una clase. BlueJ ayuda a diferenciar los distintos
	ámbitos empleando distintos colores.\\
	\\
	\textbf{Variables locales}Hasta ahora, nos hemos encontrado con dos tipos diferentes de variables: campos (variables de
	instancia) y parámetros. Ahora introducimos un tercer tipo. Lo que tienen en común todos estos
	tipos de variable es que almacenan datos, pero cada tipo de variable desempeña un papel diferente.\\
	el cuerpo de un método puede
	contener tanto declaraciones como instrucciones..\\
	Las declaraciones de variables locales parecen similares a las declaraciones de campos, pero las
	palabras clave private y public nunca aparecen en la declaración. Los constructores también
	pueden tener variables locales. Al igual que los parámetros formales, las variables locales tienen
	un ámbito que está limitado a las instrucciones del método al que pertenecen. Su tiempo de vida
	coincide con el tiempo durante el cual se está ejecutando el método: se crean cuando se invoca un
	método y se destruyen cuando el método termina.\\
	Cabe preguntarse para qué hacen faltan las variables locales si ya disponemos de campos. Las
	variables locales se usan principalmente como almacenamiento temporal, para ayudar a un método
	a completar su tarea; podemos considerarlas como un almacenamiento de datos para un único
	método. Por el contrario, los campos se utilizan para almacenar datos que permanecen durante toda
	la vida de un objeto completo. Los datos almacenados en campos son accesibles para todos los
	métodos del objeto. Tenemos que intentar evitar declarar como campos aquellas variables que solo
	tienen un uso local (en el nivel de método), es decir, cuyos valores no necesitan recordarse más allá
	de una única llamada al método. Por tanto, incluso aunque dos o más métodos de una misma clase
	utilicen variables locales con un propósito similar, no sería apropiado definirlas como campos si sus
	valores no necesitan persistir más allá del momento en que termina la ejecución de esos métodos.\\
	Al igual que los parámetros formales, las variables locales tienen
	un ámbito que está limitado a las instrucciones del método al que pertenecen. Su tiempo de vida
	coincide con el tiempo durante el cual se está ejecutando el método: se crean cuando se invoca un
	método y se destruyen cuando el método termina.\\
	\\
	\textbf{Campos parámetros y variables locales:} Los tres tipos de variables son capaces de almacenar un valor que se corresponda con su
	tipo definido. Los campos se definen fuera de los constructores y métodos. Los campos se utilizan para almacenar datos que persisten durante toda la vida de un objeto.
	Por ello, mantienen el estado actual de un objeto. Tienen un tiempo de vida que coincide con la
	duración del objeto al que pertenecen. Los campos tienen un ámbito que coincide con la clase: son accesibles desde cualquier punto de
	la clase a la que pertenecen, de modo que se pueden utilizar dentro de cualquiera de los constructores
	o métodos de la clase en la que han sido definidos.\\
	Mientras se definan como privados (private) no se podrá acceder a los campos desde ningún
	punto situado fuera de la clase en la que están definidos.\\
	Los parámetros formales y las variables locales solo persisten mientras que se está ejecutando
	un constructor o método. Su tiempo de vida coincide con la duración de una única invocación,
	por lo que sus valores se pierden entre invocaciones sucesivas. Desde ese punto de vista, actúan
	como ubicaciones de almacenamiento temporal, no permanente.\\
	Los parámetros formales se definen en la cabecera de un constructor o método. Reciben sus
	valores del exterior, siendo inicializados de acuerdo con los valores de los parámetros reales que
	forman parte de la llamada al constructor o al método. Los parámetros formales tienen un ámbito que está limitado al constructor o método en los que
	se los define.
	Las variables locales se definen dentro del cuerpo de un constructor o método. Solo pueden
	inicializarse y utilizarse dentro del cuerpo del constructor o método en el que se las define.
	Las variables locales deben inicializarse antes de poder ser utilizadas en una expresión; no se les
	proporciona un valor predeterminado.\\
	Las variables locales tienen un ámbito que está limitado al bloque en el que están definidas. No
	se puede acceder a ellas desde ningún punto situado fuera de dicho bloque.
	\section{interacción de objetos}
	\textbf{Abstracción y modulatización:} A medida que la complejidad de
	un problema aumenta, cada vez se hace más difícil controlar todos los detalles simultáneamente.\\
	La solución que usaremos para tratar con el problema de la complejidad es la abstracción. Dividiremos
	el problema en una serie de subproblemas, que a su vez dividiremos en sub-subproblemas,
	y así sucesivamente, hasta que cada problema individual sea suficientemente pequeño para poder
	resolverlo de manera sencilla. Una vez resuelto uno de los subproblemas, ya no dedicaremos más
	tiempo a pensar en los detalles de esa parte, sino que trataremos la solución como si fuera un único
	bloque componente que podemos emplear para solucionar el siguiente problema. Esta técnica se
	denomina en ocasiones divide y vencerás.\\
	la modularización y la abstracción se complementan entre sí. La modularización es el
	proceso de dividir grandes cosas (problemas) en partes más pequeñas, mientras que la abstracción
	es la capacidad de ignorar los detalles para centrarse en la panorámica general.\\
	\\
	Para ayudarnos a mantener una visión panorámica
	en los problemas complejos, tratamos de identificar subcomponentes que podamos programar
	como entidades independientes. Después, intentamos usar esos subcomponentes como si fueran
	partes simples, sin preocuparnos acerca de su complejidad interna. En la programación orientada a objetos, estos componentes y subcomponentes son precisamente
	objetos.\\
	\textbf{las clases definen tipos.} los nombres de clases pueden utilizarse como tipos.
	La declaración de un campo u otra
	variable de tipo de clase no crea automáticamente un objeto de ese tipo; al contrario, el campo
	inicialmente está vacío. El objeto asociado debe
	crearse explícitamente, y veremos cómo se hace cuando analicemos el constructor de la clase
	\subsection{Diagramas de clases y diagramas de objetos}
	El diagrama de clases
	muestra la vista estática. En él se refleja lo que tenemos en el momento de escribir el programa. (Tenemos dos clases y la flecha indica que la clase ClockDisplay hace uso de la clase Number-
	Display (NumberDisplay aparece mencionado en el código fuente de ClockDisplay). También
	vemos, por eso, que ClockDisplay depende de NumberDisplay.)\\
	el diagrama de objetos muestra la situación en tiempo de ejecución
	(cuando se está ejecutando la aplicación). Esto se denomina también vista dinámica.
	El diagrama de objetos también muestra otro detalle importante: cuando una variable almacena un
	objeto, el objeto no se almacena directamente en la variable, sino que lo que la variable contiene es
	una referencia a objeto. El objeto al que se hace referencia está almacenado
	fuera del objeto en el que aparece la referencia, y es precisamente esa referencia a objeto lo que
	enlaza los dos objetos entre sí. BlueJ proporciona solo la vista estática.\\
	Para planificar y comprender programas Java hemos de ser capaces de construir
	diagramas de objetos sobre papel o en nuestra cabeza. Cuando pensemos en lo que va a hacer
	nuestro programa, pensaremos en las estructuras de objetos que creará y en cómo interaccionarán
	esos objetos. Es esencial saber visualizar las estructuras de objetos.\\
	\\
	\textbf{Tipos primitivos y tipos de objeto:} Java trabaja con dos especies muy distintas de tipos: tipos primitivos y tipos de objeto. Los tipos primitivos
	están todos ellos predefinidos en el lenguaje Java. Los tipos de objeto son
	aquellos que están definidos mediante clases. Algunas clases se definen mediante el sistema Java
	estándar (como por ejemplo String); otras son las que escribimos nosotros mismos. \\
	Tanto los tipos primitivos como los tipos de objeto pueden emplearse como tipos, pero hay situaciones
	en las que se comportan de forma diferente. Una de las diferencias afecta al modo en
	que se almacenan los valores. Como hemos podido ver en nuestros diagramas, los valores primitivos
	se almacenan directamente en una variable. Por el contrario, los objetos no se almacenan
	directamente en la variable, sino que lo que se almacena es una referencia al objeto
	\\
	(Si uno de los operandos de una operación suma es una cadena y el otro no, entonces automáticamente
	se convierte el otro operando a una cadena, para realizar después una concatenación. Esto funciona para todos los tipos. Con independencia del tipo que se “sume” a una cadena, dicho
	tipo se convertirá automáticamente a una cadena y luego se concatenará. )\\
	\\
	La sintaxis de una operación de creación de un nuevo objeto es: \textbf{new NombreClase (lista-parámetros)}\\
	La operación new hace dos cosas:
	1 Crea un nuevo objeto de la clase indicada\\
	2 Ejecuta el constructor de dicha clase.\\
	Si el constructor de la clase se ha definido de manera que incluya parámetros, entonces habría que
	suministrar los parámetros reales en la instrucción new.\\
	\\
	Es común que las definiciones de clases contengan versiones alternativas de
	los constructores o de los métodos que proporcionan diversas formas de llevar a cabo una tarea
	concreta y que se diferencian entre sí por sus conjuntos de parámetros. Esto se conoce con el nombre
	de \textbf{sobrecarga} de un constructor o de un método.\\
	Si el método se
	encuentra dentro la misma clase que la propia llamada al método, decimos que es una llamada a
	un método interno. Las llamadas a métodos internos tienen la sintaxis
	\textbf{nombreMetodo (lista-parámetros)
	} Una llamada a un método interno no tiene nombre de variable, No se necesita una variable ya que, con una llamada a un método interno, el objeto llama
	al método de por sí.\\
	Cuando se encuentra una llamada a método, se ejecuta el método correspondiente, y luego la ejecución
	vuelve a la llamada a método y continúa con la siguiente instrucción situada después de
	la misma. Para que la signatura de un método se corresponda con la llamada a método, tanto el
	nombre como la lista de parámetros deben corresponderse. Esta necesidad de ajustarse tanto al nombre del método como a
	la lista de parámetros es importante, porque puede haber más de un método con el mismo nombre
	dentro de una clase, en caso de que ese método esté sobrecargado.\\
	Es común que las definiciones de clases contengan versiones alternativas de
	los constructores o de los métodos que proporcionan diversas formas de llevar a cabo una tarea
	concreta y que se diferencian entre sí por sus conjuntos de parámetros. Esto se conoce con el nombre
	de sobrecarga de un constructor o de un método.\\
	\\
	\textbf{llamadas a métodos externos:} La sintaxis de una llamada a método externo es:
	\textbf{objeto . nombreMetodo (lista-parámetros) } Esta sintaxis se conoce como notación con punto. Está compuesta por un nombre de objeto, un
	punto, el nombre del método y los parámetros para la llamada. Es muy importante darse cuenta de
	que lo que utilizamos aquí es el nombre de un objeto y no el nombre de una clase.\\
	La diferencia entre llamadas a métodos internos y externos está clara; la presencia de un nombre
	seguido por un punto nos informa de que el método invocado forma parte de otro objeto.\\
	El conjunto de métodos que un objeto pone a disposición de otros objetos se denomina interfaz.
	Veremos las interfaces con mucho más detalle más adelante en el libro.\\
	\\
	en ocasiones resulta ventajoso utilizar herramientas adicionales para entender mejor
	cómo se ejecuta un programa. Una de esas herramientas, a la que ahora echaremos un vistazo es
	el depurador.
	Un depurador es un programa que permite a los programadores ejecutar una aplicación paso a
	paso. Normalmente, proporciona funciones para iniciar y detener un programa en puntos seleccionados
	del código fuente, así como para examinar los valores de las variables. El nombre debugger En inglés, los errores en programas informáticos se los denomina
	coloquialmente bugs. Por ello, a los programas depuradores que ayudan a eliminar esos errores se les
	conoce con el nombre de debuggers.
	No está del todo claro de dónde proviene el término bug, que en inglés significa “insecto”. Hay una
	famosa anécdota de lo que se conoce como “el primer error informático”, que fue debido a un insecto
	real (en realidad, una polilla). Ese insecto fue encontrado, en 1945, dentro de la computadora Mark II
	por Grace Murray Hopper, una de las primeras personas que trabajó en el campo de la informática.
	Todavía se conserva en el Museo Nacional de Historia Americana del Instituto Smithsoniano un libro
	de registro en el que aparece una entrada con esta polilla pegada con cinta adhesiva al libro y con
	la anotación “Primer caso real de localización de un insecto (bug)”. Sin embargo, tal como está
	redactada esa anotación, se sugiere que el término bug ya había estado utilizándose antes de que este
	insecto real causara problemas en el Mark II.\\
	Los depuradores varían mucho en complejidad. Los utilizados por desarrolladores profesionales
	tienen una gran cantidad de funciones que resultan útiles para hacer exámenes sofisticados de múltiples
	facetas de una aplicación BlueJ tiene un depurador incorporado que es mucho más simple.
	Podemos utilizarlo para detener nuestro programa, para ejecutar el código línea por línea y para
	examinar los valores de las variables. Sin embargo, a pesar de la aparente falta de sofisticación de
	este depurador, es más que suficiente para obtener una gran cantidad de información.\\
	\\
	\textbf{La palabra clave this:} tenemos una situación que se conoce con el nombre
	de sobrecarga de nombres; es la situación en que se utiliza el mismo nombre para dos entidades
	diferentes.\\
	Es importante entender que los campos y los
	parámetros son variables diferentes, que existen independientemente unas de otras, aun cuando
	compartan nombres similares. El hecho de que un parámetro y un campo compartan un nombre no
	es ningún problema en Java.
	Lo que sí es un problema es cómo referenciar las variables para poder distinguir entre los
	dos conjuntos de valores.\\
	Si simplemente usamos el nombre de variable ¿qué variable se utilizaría, el
	parámetro o el campo?
	La especificación del lenguaje Java permite responder a esta pregunta. Especifica que se utilice
	siempre la definición que tenga su origen en el bloque circundante más próximo. todo lo que necesitamos es un mecanismo que nos permita acceder a un campo cuando
	exista una variable definida más próxima con el mismo nombre. Para esto se utiliza precisamente
	la palabra clave this. La expresión this hace referencia al objeto actual. Escribir this.from
	hace referencia al campo from del objeto actual. Por tanto, esta estructura nos da un medio de
	referirnos al campo, en lugar de al parámetro que tiene el mismo nombre.\\
	this.from = from;
	Esta instrucción, como ahora vemos, tiene el siguiente efecto:
	campo denominado from = parámetro denominado from;
	En otras palabras, asigna el valor del parámetro al campo que tiene el mismo nombre. Si un cierto nombre describe perfectamente el uso,
	resulta razonable emplearlo en ambos casos y aceptar la complicación de utilizar la palabra clave
	this dentro de la asignación para resolver el conflicto de nombres.\\
	El depurador no solo nos permite interrumpir la ejecución del programa e inspeccionar las variables,
	sino que también nos permite avanzar en la ejecución lentamente. Al estar parados en un punto de interrupción, si hacemos clic en el botón Step (paso) se ejecuta una
	única línea de código y luego la ejecución vuelve a detenerse.\\
	(Recuerde: construir un objeto hace
	dos cosas, crear un objeto y ejecutar el constructor). Invocar el constructor funciona de forma muy
	similar a la invocación de métodos
	\section{agrupación de objeos}
	El principal objeto de este capítulo es presentar algunas de las formas en las que pueden agruparse
	objetos para formar colecciones. En particular, hablaremos de la clase ArrayList como ejemplo
	de colecciones de tamaño flexible. Estrechamente asociada con las colecciones se encuentra la
	necesidad de iterar a lo largo de los elementos que esas colecciones contienen. Con ese propósito,
	presentaremos tres nuevas estructuras de control: dos versiones del bucle for y el bucle while.\\
	\\
	Vimos entonces
	que la abstracción nos permite simplificar un problema, identificando componentes discretos que
	puedan contemplarse como un todo, en lugar de tener que preocuparnos por sus detalles. Analizaremos
	este principio en acción cuando comencemos a hacer uso de las clases de librería disponibles
	en Java. Aunque estas clases no son, estrictamente hablando, parte del lenguaje, algunas de
	ellas están íntimamente asociadas con la escritura de la mayor parte de los programas Java, por lo
	que a menudo se piensa en ellas como en una parte más del lenguaje. La mayoría de la gente que
	escribe programas Java comprueba constantemente las librerías para ver si alguien ha escrito ya
	una clase que ellos puedan aprovechar. De esta forma, se ahorran una gran cantidad de esfuerzo,
	que puede emplearse mejor en trabajar en otras partes del programa. El mismo principio se aplica en la mayoría de los demás lenguajes de programación, que también suelen disponer de librerías
	de clases útiles. Por tanto, merece la pena familiarizarse con el contenido de la librería y saber
	cómo usar las clases más comunes. La potencia de la abstracción reside en que, para usar una clase
	de manera efectiva, normalmente no nos hace falta conocer muchos detalles (¡de hecho, acaso
	ninguno!) acerca de las interioridades de la clase.
	Si utilizamos una clase de la librería, lo que haremos será escribir código que cree instancias de
	esa clase, después de lo cual nuestros objetos podrán interactuar con los objetos de la librería.\\
	\textbf{la colección como abstracción:} la idea de colección, el concepto
	de agrupar cosas para poder referirnos a ellas y manejarlas de manera conjunta. En un contexto de programación, la abstracción colección se convierte en una clase de un
	cierto tipo, y las operaciones serían los métodos de esa clase. Una colección (mi colección de
	música) sería una instancia de la clase. Además, los elementos almacenados en una instancia
	de colección serían, ellos mismos, objetos.\\
	Hasta ahora, no hemos visto ninguna característica de Java que nos permita agrupar un número
	arbitrario de elementos. Quizá podríamos definir una clase con un grupo muy extenso de campos
	individuales, para abarcar un número fijo, pero muy elevado, de elementos. Sin embargo, los
	programas suelen necesitar una solución que sea más general. Sería adecuada aquella que no nos
	exigiera saber de antemano cuántos elementos vamos a agrupar, y que no nos obligara a fijar un
	límite superior para dicho número. la forma más simple posible de agrupar objetos, mediante una lista secuencial desordenada
	de tamaño flexible: ArrayList.\\
	\\
	Vamos a escribir una clase que nos ayude a organizar nuestros archivos de música almacenados
	en una computadora. Nuestra clase no almacenará en realidad los detalles de los archivos; en su
	lugar, lo que hará será delegar esa responsabilidad en la clase estándar de librería ArrayList, que
	nos ahorrará mucho trabajo. Entonces, ¿por qué necesitamos escribir una clase? Un punto importante
	que hay que tener en mente al tratar con las clases de librería es que estas no se han escrito
	para ningún escenario de aplicación concreto, son clases de propósito general. Esto significa que quienes proporcionan las
	operaciones específicas de cada escenario son las clases que escribamos para utilizar las clases de
	librería.\\
	Librerías de clases Una de las características de los lenguajes orientados a objetos que les dota
	de más potencia es que a menudo suelen estar acompañados por librerías de clases. Estas librerías
	suelen contener varios cientos o miles de clases distintas, que han demostrado ser útiles para los
	desarrolladores en un amplio rango de proyectos distintos. Java denomina a sus librerías paquetes. Las
	clases de librería se emplean exactamente de la misma forma que utilizaríamos nuestras clases. Las
	instancias se construyen utilizando new y las clases tienen campos, constructores y métodos\\
	\\
	La primera línea del archivo de clase ilustra la forma en la que obtenemos acceso a una clase de
	librería en Java: mediante una instrucción de importación (import):
	import java.util.ArrayList;\\
	Esto hace que la clase ArrayList del paquete java.util esté disponible a la hora de definir
	nuestra clase. Las instrucciones de importación deben colocarse siempre antes de las instrucciones
	de clases en un archivo. Una vez importado desde un paquete de esta manera un archivo de
	clase, podemos utilizar dicha clase como si fuera una de nuestras propias clases.\\
	cuando señalamos que ArrayList es una clase de colección de propósito general,
	es decir, que no está restringida en lo que respecta a los tipos de objeto que puede almacenar.
	Sin embargo, cuando creamos un objeto ArrayList, tenemos que ser específicos acerca
	del tipo de objetos que se almacenarán en esa instancia concreta. Podemos almacenar cualquier
	tipo que decidamos, pero es necesario designar dicho tipo al declarar una variable ArrayList.
	Las clases como ArrayList, que se parametrizan con un segundo tipo, se denominan clases
	genéricas.\\
	Al utilizar colecciones, por tanto, siempre tenemos que especificar dos tipos: el tipo de la propia
	colección (en este caso, ArrayList) y el tipo de los elementos que pretendemos almacenar
	en esa colección (que aquí es String). Podemos leer la definición completa de tipo ArrayList$<String>$ como “una colección ArrayList de objetos de tipo String”.\\
	Observe que al crear la instancia ArrayList hemos escrito la siguiente instrucción:
	files = new ArrayList$<String>$(); Se trata de la denominada notación diamante (debido a que los dos corchetes en ángulo crean una
	forma de rombo) y parece poco habitual . Antes hemos visto que la instrucción new adopta la forma
	siguiente:
	new nombre-tipo (parámetros)\\
	La primera versión, con la notación diamante (que
	deja la mención del tipo String) es simplemente un modo abreviado utilizado por comodidad.
	Si la creación del objeto colección se combina con una asignación, el ordenador podrá deducir el
	tipo de los elementos de colección a partir del tipo de variable del lado izquierdo de la asignación,
	y Java nos permite no tener que definirla de nuevo. El tipo de elemento se infiere automáticamente
	a partir del tipo de variable.\\
	La clase ArrayList define numerosos métodos, pero nosotros, para dar soporte a la funcionalidad
	que nos hace falta, por el momento solo utilizaremos cuatro de ellos: add, size, get y remove.\\
	\\
	\textbf{Estructuras de objetos con colecciones:} Para entender cómo opera un objeto colección como ArrayList resulta útil examinar un diagrama
	de objetos. Existen al menos tres características de la clase ArrayList que debemos considerar:
	\\- Es capaz de incrementar su capacidad interna según sea necesario: a medida que se añaden
	nuevos elementos, se limita a crear espacio para ellos.
	\\- Mantiene su propio contador privado, el número de elementos que almacena en cada instante.
	Su método size devuelve el valor de ese contador.
	\\- Mantiene el orden de los elementos que se inserten en la lista. El método add almacena cada
	nuevo elemento al final de la lista. Posteriormente, podemos extraerlos en el mismo orden.\\
	Todo el trabajo complicado se realiza
	dentro del objeto ArrayList. Esta es una de las grandes ventajas de utilizar clases de librería.
	Alguien ha invertido tiempo y esfuerzo para implementar algo útil y nosotros tenemos acceso a esa
	funcionalidad de manera prácticamente gratuita sin más que utilizar esa clase.
	Por el momento, no necesitamos preocuparnos por cómo es capaz un ArrayList de soportar
	esas características. Nos basta con ser capaces de apreciar lo útil que es esta capacidad. Recuerde:
	esta supresión de los detalles es uno de los beneficios que la abstracción nos proporciona; implica
	que podemos utilizar ArrayList para escribir cualquier número de clases distintas que necesiten
	almacenar un número arbitrario de objetos.\\
	\\
	La nueva notación que utiliza los corchetes angulares merece alguna explicación adicional. El tipo
	de nuestro campo files se ha declarado como
	ArrayList$<String>$ La clase que utilizamos aquí se llama simplemente ArrayList, pero requiere que se especifique
	un segundo tipo como parámetro cuando se usa para declarar campos u otras variables. Las clases
	que requieren un parámetro de tipo como este se denominan clases genéricas. A diferencia de
	otras clases que hemos visto hasta ahora, las clases genéricas no definen un único tipo en Java,
	sino que pueden definir muchos tipos.\\
	Cada ArrayList particular es un tipo distinto, que puede utilizarse
	en las declaraciones de campos, parámetros y valores de retorno.\\
	los elementos
	almacenados en las colecciones ArrayList tienen una numeración o posicionamiento implícito
	que comienza en 0. La posición de un objeto dentro de una colección se conoce comúnmente como
	el nombre de índice. Al primer elemento añadido a una colección se le da el número de índice 0, al
	segundo se le da el número de índice 1, y así sucesivamente. Los métodos listFile y removeFile ilustran la forma en que se utiliza el número de índice
	para acceder a un elemento de un ArrayList: uno lo hace a través del método get y el otro a través
	del método remove.\\
	La clase ArrayList tiene un
	método remove que toma como parámetro el índice del objeto que hay que eliminar. Un detalle
	del proceso de eliminación del que debemos ser conscientes es que puede cambiar los valores de
	índice en los que están almacenados otros objetos de la colección. Si se elimina un objeto con un
	número de índice bajo, entonces la colección desplazará una posición todos los elementos posteriores
	con el fin de rellenar el hueco. Como consecuencia, sus números de índice disminuirán en
	una unidad. es posible insertar elementos en un ArrayList en una posición
	distinta del final de la lista. Esto quiere decir que los elementos que ya se encuentren en la
	lista podrían ver sus números de índice incrementados cuando se añada un nuevo elemento.\\
	\\
	\textbf{bucles:} el uso de instrucciones de bucle, que también se
	conocen con el nombre de estructuras iterativas de control. El primer bucle que presentaremos para enumerar los archivos es un bucle especial que se utiliza
	con colecciones y que elimina completamente la necesidad de utilizar una variable de índice: se
	denomina bucle for-each. Un bucle for-each es una de las formas de llevar a cabo repetidamente un conjunto de acciones
	sobre los elementos de una colección\\
	for(TipoElemento elemento : colección) \{
		cuerpo del bucle
	\}\\
	El principal elemento nuevo de Java es la palabra for. El lenguaje Java tiene dos variantes del
	bucle for: una es el bucle for-each, que es el que estamos analizando aquí, y la otra se denomina
	simplemente bucle for.\\
	Un bucle for-each tiene dos partes: una cabecera del bucle (la primera línea de la instrucción de
	bucle) y un cuerpo de bucle situado a continuación de la cabecera. El cuerpo contiene aquellas
	instrucciones que queremos ejecutar una y otra vez. El bucle for-each obtiene su nombre de la forma en que podemos interpretar su sintaxis: si leemos
	la palabra clave for como “for each” (“para cada”) y los dos puntos de la cabecera del bucle como
	“in” (“en”). Analicemos el bucle con un poco más de detalle. La palabra clave for inicia el bucle. Va seguida
	de una pareja de paréntesis, dentro de los cuales se definen los detalles del bucle. El primero de
	esos detalles es la declaración
	que declara una nueva variable local que se utilizará para almacenar sucesivamente los distintos elementos de la lista. A esta variable
	la denominamos variable de bucle. Podemos elegir el nombre que queramos para esta variable,
	al igual que sucede con cualquier otra; El tipo de la
	variable de bucle debe coincidir con el tipo de elemento declarado para la colección que vayamos
	a utilizar.\\
	A continuación, aparece un carácter de dos puntos y luego la variable que contiene la colección
	que queremos procesar. Para esta colección, cada elemento será asignado por turno a la variable de
	bucle; y para cada una de esas asignaciones, se ejecuta una vez el cuerpo del bucle. En el cuerpo
	del bucle, utilizamos la variable de bucle para hacer referencia a cada elemento. \\
	El bucle for-each se emplea siempre para iterar a través de una colección. Nos proporciona una
	forma de acceder a cada elemento de la colección sucesivamente, uno por uno, y procesar esos
	elementos de la forma que deseemos. Podemos decidir llevar a cabo las mismas acciones con cada elemento (como hicimos al imprimir la lista completa) o podemos ser selectivos y filtrar la lista
	(como hicimos al imprimir solo un subconjunto de la colección). El cuerpo del bucle puede ser
	todo lo complicado que queramos.\\
	No obstante, esta sencillez esencial lleva aparejadas necesariamente algunas limitaciones. Por
	ejemplo, una restricción es que no podemos modificar lo que está almacenado en la colección
	mientras iteramos a través de ella: ni añadir nuevos elementos ni eliminar elementos de la colección.
	Sin embargo, esto no significa que no podamos cambiar el estado de los objetos que ya están
	dentro de la colección También hemos visto que el bucle for-each no nos proporciona un valor de índice para los elementos
	de la colección. Si deseamos uno, tendremos que declarar y mantener nuestra propia variable
	local. La razón tiene que ver de nuevo con la abstracción. Al tratar con colecciones e iterar a través
	de ellas, resulta útil tener presentes dos consideraciones:\\
	Un bucle for-each proporciona una estructura general de control para iterar a través de diferentes
	tipos de colecciones.\\
	Existen algunos de tipos de colecciones que no asocian de manera natural índices enteros con
	los elementos que almacenan\\
	Por tanto, el bucle for-each abstrae la tarea de procesar una colección completa elemento a elemento
	y es capaz de manejar diferentes tipos de colección. No necesitamos saber los detalles de
	cómo manipula las colecciones.\\
	\\
	le recomendamos emplear un bucle for-each solo si está seguro de querer procesar la
	colección completa. Dicho de otra forma, una vez que el bucle comience, sabremos con seguridad
	cuántas veces se va a ejecutar el cuerpo del bucle; ese número de veces será igual al tamaño de la
	colección. Este estilo se denomina en ocasiones iteración definida. Para aquellas tareas en las que
	queramos detener anticipadamente el procesamiento de la colección hay otros bucles más apropiados
	que se pueden utilizar como, por ejemplo, el bucle while, que presentaremos a continuación.
	En estos casos, el número de veces que se ejecutará el cuerpo del bucle es menos preciso; normalmente,
	dependerá de lo que suceda durante la iteración. Este estilo se denomina en ocasiones
	iteración indefinida. Un bucle for-each proporciona una iteración definida;
	dado el estado de una colección concreta, el cuerpo del bucle se ejecutará un número de veces que se corresponde exactamente con el tamaño de dicha colección. Ahora bien, hay muchas situaciones
	en las que queremos repetir una serie de acciones pero en las que no podemos predecir de
	antemano exactamente cuántas veces será. Un bucle for-each no nos sirve de ayuda en estos casos \\
	\\
	\textbf{ieración indefinida:} la acción se repetirá
	un número de veces no predecible hasta que se complete la tarea. frecuentemente nos encontraremos con situaciones en las que querremos
	seguir haciendo una determinada cosa hasta que la repetición deje de ser necesaria. De hecho,
	estas situaciones son tan comunes que la mayoría de los lenguajes de programación proporcionan
	al menos una (y normalmente más de una) estructura de bucle para expresarlas. Un bucle while consta de una cabecera y de un cuerpo; el cuerpo está pensado para ser ejecutado
	de manera repetida. He aquí la estructura de un bucle while donde condición booleana y cuerpo
	del bucle son pseudocódigo, pero todo lo restante es la sintaxis de Java:\\
	while (condición booleana) \{\\
		cuerpo del bucle\\
	\}\\
	El bucle se inicia con la palabra clave while, seguida de una condición booleana. La condición es
	la que controla, en último término, cuántas veces se iterará un bucle concreto. La condición se evalúa
	cuando el control del programa alcanza por primera vez el bucle, y vuelve a evaluarse después
	de ejecutar cada vez el cuerpo del bucle. Esto es lo que da al bucle while su carácter indefinido:
	ese proceso de reevaluación. Si la condición se evalúa como true, entonces se ejecuta el cuerpo del
	bucle; y una vez que la condición se evalúe como false, se da por terminada la iteración. Entonces
	el programa se salta el cuerpo del bucle y la ejecución continúa con lo que haya a continuación
	del bucle. \\
	ventajas: en primer lugar, el bucle while no necesita estar relacionado con una colección
	(podemos construir un bucle con cualquier condición que podamos escribir en forma de expresión
	booleana); en segundo lugar, aunque utilicemos el bucle para procesar una colección, es posible
	que no necesitemos procesar todos los elementos; en lugar de ello, podríamos detenernos anticipadamente,
	si así lo deseamos, e incluir otra componente dentro de la condición del bucle que
	exprese por qué querríamos terminar el bucle. Por supuesto, estrictamente hablando, lo que la
	condición del bucle expresa en realidad es por qué querríamos continuar, y es la negación de
	esa condición la que hace que el bucle se detenga.
	Una ventaja de tener una variable de índice explícita es que podemos utilizar su valor tanto dentro
	como fuera del bucle, lo que no podíamos hacer en los ejemplos de for-each. Tener una variable de índice local puede ser muy importante al realizar búsquedas en una lista,
	porque puede proporcionar información sobre dónde estaba ubicado el elemento y podemos hacer
	que esa información siga estando disponible una vez que el bucle haya finalizado.
	\\
	\\
	\textbf{búsquedas:} La característica clave de una búsqueda es que implica una iteración indefinida; así tiene que ser
	necesariamente, porque si supiéramos exactamente dónde buscar, no nos haría falta realizar ninguna
	búsqueda. En lugar de ello, lo que tenemos que hacer es iniciar una búsqueda y luego nos
	hará falta un número desconocido de iteraciones antes de completarla. Esto implica que un bucle for-each es inapropiado para las búsquedas, porque siempre llevará a cabo su conjunto completo
	de iteracione En situaciones reales de búsqueda, tenemos que tener en cuenta que la búsqueda puede fallar: es
	posible que nos quedemos sin lugares en los que buscar. Eso quiere decir que normalmente tendremos
	que tomar en consideración dos posibilidades de finalización a la hora de escribir un bucle
	de búsqueda: -La búsqueda tiene éxito después de un número indefinido de iteraciones.\\
	-La búsqueda falla después de agotar todas las posibilidades.\\
	\\Debemos tener en cuenta las dos posibilidades a la hora de escribir la condición del bucle. Como
	la condición del bucle debe evaluarse como true si queremos iterar otra vez más, cada uno de los
	criterios de finalización debe poder hacer, por sí mismo, que la condición se evalúe como false
	para detener el bucle. \\
	El hecho de que terminemos por analizar la lista completa en aquellos casos en los que la búsqueda
	falla no hace que las búsquedas fallidas constituyan un ejemplo de iteración definida. La característica
	clave de la iteración definida es que podemos determinar el número de iteraciones en el
	momento de iniciarse el bucle. Ese no será nunca el caso cuando estemos haciendo una búsqueda.\\
	\\
	También necesitamos añadir una segunda parte a la condición que indique si hemos encontrado
	ya el elemento de búsqueda y que detenga la búsqueda en caso de que lo hayamos hecho.\\
	Una variable denominada searching (o, por ejemplo, missing) configurada inicialmente con
	el valor true haría que la búsqueda continuara hasta que la variable se configurara como false
	dentro del bucle, después de haber encontrado el elemento. Una variable denominada found, configurada inicialmente como false y utilizada en la condición
	como !found haría que la búsqueda continuara hasta que la variable se configurara como
	true después de encontrar el elemento. Recuerde que toda la condición debe evaluarse como true si queremos continuar buscando, y que
	debe evaluarse como false si queremos dejar de buscar, por la razón que sea.\\
	\\
	Los bucles no se utilizan solo con colecciones. Existen muchas situaciones en las que queremos
	repetir un bloque de instrucciones en el que no interviene para nada ninguna colección. Los bucles for-each solo se pueden usar para iterar a través de colecciones.\\
	\\
	Una de las ventajas de la orientación a objetos es que nos permite diseñar clases que modelen
	bastante fielmente los comportamientos y estructura inherentes de las entidades del mundo real
	que estemos intentando representar. Esto se consigue escribiendo clases cuyos campos y métodos
	se correspondan con los de los atributos\\
	\\
	\textbf{El tipo iterador:} Utiliza un bucle
	while para realizar la iteración y un objeto Iterator en lugar de una variable de índice entera para
	controlar la posición dentro de la lista. Tenemos que ser muy cuidadosos con la denominación en
	este punto, porque Iterator (observe la mayúscula I) es un tipo de Java, pero también nos encontraremos
	con un método denominado iterator (observe la minúscula i).\\
	Examinar todos los elementos de una colección es tan común, que ya hemos visto que existe un
	estructura de control especial (el bucle for-each) que está diseñada a propósito para esta tarea. Además,
	las distintas clases de librería para colecciones de Java proporcionan un tipo común diseñado
	a medida para soportar la iteración, y ArrayList es típica a este respecto.
	El método iterator de ArrayList devuelve un objeto Iterator. Iterator también está
	definido en el paquete java.util, así que debemos añadir una segunda instrucción de importación
	al archivo de clase para poder utilizarlo:\\
	\textbf{import java.util.ArrayList;}\\
	\textbf{import java.util.Iterator;}\\
	Un Iterator proporciona simplemente tres métodos, y dos de ellos se usan para iterar a través de
	una colección: hasNext y next. Ninguno de ellos admite parámetros, pero ambos tienen tipos
	de retorno definidos, de modo que se usan en expresiones.\\
	La forma en que normalmente utilizamos
	un Iterator puede describirse en pseudocódigo como sigue:\\
	\\
	Iterator $<TipoElemento>$ it = myCollection.iterator();\\
	while(it.hasNext())\{\\
	\textit{llamar} it.next() \textit{para obtener el siguiente elemento}\\
	\textit{hacer algo con ese elemento}
	\}\\
	\\
	utilizamos primero el método iterator de la clase ArrayList
	para obtener un objeto Iterator. Observe que Iterator es también un tipo genérico, por lo que
	lo parametrizamos con el tipo de los elementos contenidos en la colección a través de la cual estamos
	iterando. A continuación, empleamos ese Iterator para comprobar repetidamente si hay
	más elementos, mediante it.hasNext(), y para obtener el siguiente elemento, mediante it.next().
	Un punto importante que hay que resaltar es que es al objeto Iterator al que le pedimos que
	devuelva el siguiente elemento y no al objeto colección. De hecho, tendemos a no referirnos directamente
	en absoluto a la colección dentro del cuerpo del bucle; toda la interacción con la colección
	se realiza a través del Iterator.\\
	Un aspecto
	concreto que hay que resaltar acerca de esta última versión es que utilizamos un bucle while, pero
	no necesitamos preocuparnos de la variable index. Esto se debe a que Iterator controla el
	punto en el que nos encontramos dentro de la colección, de modo que sabe si quedan más elementos
	(hasNext) y qué elemento devolver (next) si es que todavía quedan.\\
	\\
	Una de las claves para comprender cómo funciona Iterator es que la llamada a next hace que
	el objeto Iterator devuelva el siguiente elemento de la colección y luego avance más allá de ese
	elemento. Por tanto, las llamadas sucesivas a next en un Iterator siempre devolverán elementos
	diferentes; no se puede volver al elemento anterior después de haber invocado next. En algún
	momento, el Iterator alcanzará el final de la colección y devolverá false al hacerse una llamada
	a hasNext. Una vez que hasNext ha devuelto false, sería un error tratar de invocar next sobre
	ese objeto Iterator concreto; de hecho, el objeto Iterator habrá sido “agotado” y ya no tendrá
	ninguna utilidad.
	\\
	\\
	\textbf{Eliminación de elemnetos:}\\
	Si intentamos
	modificar la colección utilizando uno de los métodos remove de la misma mientras estamos en
	mitad de una iteración, el sistema nos dará un error (denominado ConcurrentModification-
	Exception). Sucede así porque cambiar la colección en mitad de una iteración tiene el potencial
	de confundir la situación enormemente. por lo que simplemente se prohíbe la utilización del método remove de la colección
	durante una iteración del bucle for-each.\\
	La solución apropiada para efectuar la eliminación mientras estamos iterando consiste en utilizar
	un Iterator. Su tercer método (además de hasNext y next) es remove. No admite ningún
	parámetro y tiene un tipo de retorno void. Invocar remove hará que sea eliminado el elemento
	devuelto por la llamada más reciente a next.\\
	Iterator $<Track>$ it = tracks.iterator();\\while(it.hasNext())\{\\
	Track t = it.next();\\
	String artist = t.getArtist();\\
	if (artist.equals(artistToRemove)) \{\\
	it.remove();\\
	\}\\
	\}	\\
	De nuevo, observe que no usamos la variable de colección tracks en el cuerpo del bucle. Aunque
	tanto ArrayList como Iterator tienen métodos remove, debemos utilizar el método remove
	de Iterator, no el de ArrayList.
	Utilizar el método remove de Iterator es menos flexible: no podemos eliminar elementos arbitrarios,
	sino solo el último elemento extraído por el método next de Iterator. Por otro lado, sí
	que se permite utilizar el método remove de Iterator durante una iteración. Dado que el propio
	Iterator está informado de la eliminación (y se encarga de llevarla a cabo por nosotros), puede
	mantener apropiadamente la iteración sincronizada con la colección.
	Dicha eliminación no es posible con el bucle for-each, porque no disponemos ahí de un Iterator
	con el que trabajar.\\ En este caso, necesitamos utilizar el\textbf{ bucle while con un Iterator.}
	\\
	Técnicamente, también podemos eliminar elementos usando el método get de la colección con un
	índice para la iteración. Sin embargo, no le recomendamos que haga esto, porque los índices de los
	elementos pueden cambiar cuando añadimos o quitamos elementos y es muy fácil conseguir que
	la iteración opere con índices incorrectos, cuando modificamos la colección durante la iteración.
	El uso de un Iterator nos protege frente a tales errores.\\
	\\
	(hemos visto cómo podemos utilizar un objeto ArrayList, creado a
	partir de una clase de la librería de clases, para almacenar un número arbitrario de objetos dentro
	de una colección. No tenemos que decidir de antemano cuántos objetos vamos a almacenar y el
	objeto ArrayList lleva automáticamente a la cuenta del número de elementos que almacena.)\\
	\\
	Un principio muy importante es que, si una variable contiene el valor null, no se debe realizar
	ninguna llamada a método con ella. La razón debería estar clara: como los métodos pertenecen a
	objetos, no podemos invocar un método si la variable no hace referencia a un objeto. Esto quiere
	decir que en ocasiones tenemos que usar una instrucción if para comprobar si una variable contiene
	null o no antes de invocar un método sobre dicha variable. Si no se hace esta comprobación,
	se obtendrá el error de tiempo de ejecución NullPointerException, que es muy común.\\
	\\
	\textbf{objetos anónimos}\\
	El método enterLot de Auction ilustra un concepto bastante común: los objetos anónimos.
	Podemos ver esto en la siguiente instrucción:\\
	lots.add(new Lot(nextLotNumber, description));\\
	Aquí estamos haciendo dos cosas:\\
	-Estamos creando un nuevo objeto Lot.\\
	-También estamos pasando este nuevo objeto al método add de ArrayList.\\
	lo que hacemos es crear un objeto anónimo, un objeto sin nombre, y pasárselo directamente
	al método que va a utilizarlo.
	\\
	\\
	\textbf{Utilización de colecciones:}\\
	La clase de colección ArrayList (y otras como ella) constituye una herramienta de programación
	importante, porque muchos problemas de programación implican trabajar con colecciones
	de objetos de tamaño variable.
	\section{comportamientos más sofisticados}
	\textbf{Documentación para clases de librería:} La librería Java estándar es enorme. Está formada por miles de clases, cada una de las cuales tiene
	muchos métodos, que a su vez pueden tener o no parámetros, contener o no tipos de retorno. Es
	imposible memorizar todos los métodos y los detalles correspondientes a cada uno. En lugar de
	ello, lo que un buen programador Java debe hacer es:\\
	-Conocer por su nombre algunas de las clases más importantes y sus métodos (ArrayList es
	una de esas clases importantes).\\
	-Saber localizar información acerca de esas clases y buscar los correspondientes detalles (como,
	por ejemplo, métodos y parámetros).
	\\
	startWith es un método dentro de la clase String, lee si el estring empieza con tales caracteres que introduzcas. La clase String es una de las clases de la librería estándar de clases Java. Podemos conocer más
	detalles acerca de la misma leyendo la documentación de librería para la clase String.
	Para ello, seleccione el elemento Java Class Libraries del menú Help de BlueJ. Se abrirá un explorador
	web mostrando la página principal de la documentación de la API (Application Programming
	Interface, Interfaz de programación de aplicaciones) de Java\\
	Verá que la documentación incluye diferentes elementos de información. Entre otros, se incluyen
	los siguientes:\\
	-El nombre de la clase.\\
	-Una descripción general del propósito de la clase.\\
	-Una lista de los constructores y métodos de la clase.\\
	-Los parámetros y los tipos de retorno para cada constructor y método.\\
	-Una descripción del propósito de cada constructor y método.\\
	Esta información, tomada conjuntamente, se denomina interfaz de una clase. Observe que la interfaz
	no muestra el código fuente que implementa la clase. Si una clase está bien descrita (es decir,
	si su interfaz está bien escrita), entonces el programador no necesita ver el código fuente para ser
	capaz de utilizar la clase. Con ver la interfaz, tenemos toda la información necesaria. Esto es de
	nuevo un ejemplo del concepto de abstracción.\\
	\\
	El código fuente subyacente, que es el que hace que la clase funcione, se conoce como implementación
	de la clase. Normalmente, un programador trabaja en la implementación de una clase a la
	vez que hace uso de otras diversas clases a través de sus interfaces.
	Esta distinción entre la interfaz y la implementación es un concepto muy importante, que volverá
	a aparecer una y otra en este capítulo y en capítulos posteriores.\\
	La interfaz de un método consta de la signatura del método y de un comentario (mostrado aquí en
	cursiva). La signatura de un método incluye (en este orden):\\
	-Un modificador de acceso (que aquí es public), del que hablaremos más adelante.\\
	-El tipo de retorno del método (en este caso int).\\
	-El nombre del método.\\
	-Una lista de parámetros (que en este ejemplo está vacía); el nombre y los parámetros reciben
	también conjuntamente el nombre de signatura del método.\\
	La interfaz proporciona todo lo que necesitamos conocer para hacer uso de este método. La documentación de la clase String nos dice que dispone de un método denominado trim para
	eliminar espacios al principio y al final de la cadena de caracteres.\\
	Un detalle importante acerca de los objetos String es que son inmutables; es decir, no pueden
	modificarse después de haberlos creado. Fíjese especialmente en que el método trim, por ejemplo,
	devuelve una nueva cadena de caracteres, no modifica la cadena original. Preste especial atención
	al siguiente comentario de “Error común”.\\
	Error común Es un error común en Java tratar de modificar una cadena. Por ejemplo, escribiendo\\
	input.toUpperCase();\\
	Esto es incorrecto (las cadenas de caracteres no pueden modificarse), aunque lamentablemente no
	produce ningún error. La instrucción simplemente no tienen ningún efecto, y la cadena de entrada no
	será modificada.
	El método toUpperCase, así como otros métodos de cadena, no modifica la cadena original, sino
	que devuelve una nueva cadena que es similar a la original, pero con algunos cambios aplicados
	(en este caso, los caracteres se han pasado a mayúscula). Si queremos modificar nuestra variable
	de entrada, entonces tenemos que asignar otra vez este nuevo objeto a la variable (descartando la
	original), como en el siguiente ejemplo:\\
	input = input.toUpperCase();\\
	El nuevo objeto también podría asignarse a otra variable o procesarse de alguna otra manera.\\
	\\
	Después de estudiar la interfaz del método trim, podemos ver que se pueden eliminar los espacios
	de una cadena de entrada con la siguiente línea de código:\\
	input = input.trim();
	\\
	Este código solicitará al objeto String almacenado en la variable input que cree una nueva
	cadena, similar a la anterior, pero sin los espacios iniciales y finales. El nuevo objeto String se
	almacena entonces en la variable input, porque no tenemos ningún uso adicional que dar a la
	cadena de caracteres anterior. Por tanto, después de esta línea de código, input hace referencia a
	una cadena que no tiene espacios ni al principio ni al final.\\
	\textbf{Comporbación de la Igualdad entre cadenas}
	El operador de igualdad (==) comprueba si cada lado del operador hace referencia
	al mismo objeto, no si tienen el mismo valor. Son dos cosas completamente distintas.\\
	La solución es utilizar el método \textbf{equals}, definido en la clase String. Este método comprueba
	correctamente si el contenido de dos objetos String coincide.\\
	\\
	\textbf{Adición de comportamiento aleatorio}\\
	 Aleatorio y seudoaleatorio La generación de números aleatorios en una computadora no es tan
	 fácil de realizar, de hecho, como inicialmente podría pensarse. Dado que las computadoras operan
	 de una forma bien definida y determinista, que descansa en el hecho de que todos los cálculos son
	 predecibles y repetibles, proporcionan poco espacio para un comportamiento realmente aleatorio.
	 Los investigadores han propuesto, a lo largo del tiempo, muchos algoritmos para generar secuencias
	 de números aparentemente aleatorias. Estos números normalmente no son realmente aleatorios, sino
	 que se generan de acuerdo con una serie muy complicada de reglas. Por ello se los denomina números
	 seudoaleatorios.
	 En un lenguaje como Java, la generación de números seudoaleatorios está implementada,
	 afortunadamente, en una clase de librería, por lo que lo único que tenemos que hacer para obtener un
	 número seudoaleatorio es hacer algunas llamadas a la librería.\\
	 \textsc{La clase Random:}\\
	 Para generar un número aleatorio, tenemos que:\\
	 -Crear una instancia de la clase Random.\\
	 -Hacer una llamada a un método de dicha instancia para obtener un número.\\
	 Al examinar la documentación, vemos que hay varios métodos denominados nextAlgo para generar
	 valores aleatorios de distintos tipos. El que genera un número aleatorio entero se denomina
	 nextInt.\\
	 El siguiente fragmento ilustra el código necesario para generar e imprimir un número aleatorio
	 entero:\\
	 Random randomGenerator;\\
	 randomGenerator = new Random();\\
	 int index = randomGenerator.nextInt();\\
	 System.out.println(index);\\
	Este fragmento de código crea una nueva instancia de la clase Random y la almacena en la variable
	randomGenerator. A continuación, llama al método nextInt para recibir un número aleatorio,
	lo almacena en la variable index y al final lo imprime. Su clase solo debe crear un instancia de la clase Random (en su constructor) y almacenarla en un
	campo. No cree una nueva instancia de Random cada vez que desee generar un nuevo número.\\
	\\
	Números aleatorios con rango limitado:\\
	Los números aleatorios que hemos visto hasta ahora se generaban a partir del rango completo
	de enteros Java (–2147483648 a 2147483647). Eso está bien para un experimento, pero rara vez
	resulta útil. Más frecuentemente, lo que querremos es obtener números aleatorios dentro de un
	rango limitado específico.\\
	La clase Random también ofrece un método para satisfacer esta necesidad. Se llama nextInt,
	pero tiene un parámetro para especificar el rango de números que nos gustaría usar.\\
	Al utilizar un método que genere números aleatorios a partir de un rango especificado, hay que
	tener cuidado de comprobar si los límites son inclusivos o exclusivos. El método nextInt (int n)
	en la clase Random de la librería Java, por ejemplo, especifica que genera un número comprendido entre 0 (inclusive) y n (exclusive). Esto significa que el valor 0 está incluido en los posibles resultados,
	mientras que el valor especificado para n no lo está. El número más alto que puede devolver
	una de esas llamadas es n–1.\\
	\\
	Generación de respuestas aleatorias:\\
	-Declarar un campo de tipo Random para almacenar el generador de números aleatorios.\\
	-Declarar un campo de tipo ArrayList para almacenar nuestras posibles respuestas.\\
	-Crear los objetos Random y ArrayList en el constructor de Responder.\\
	-Rellenar la lista de respuestas con algunas frases.\\
	-Seleccionar y devolver una frase aleatoria cuando se invoque generateResponse.\\
	public String generateResponse()\{\\
	int index = randomGenerator.nextInt(responses.size());\\
	return responses.get(index);\\
	\}	\\
	La primera línea del código en este método hace tres cosas:\\
	-Calcula el tamaño de la lista de respuestas llamando a su método size.\\
	-Calcula el tamaño de la lista de respuestas llamando a su método size.\\
	-Almacena ese número aleatorio en la variable local index.\\
	Es importante observar que este segmento de código generará un número aleatorio en el rango
	de 0 a listSize–1 (inclusive). Esto encaja perfectamente con los índices legales para un
	ArrayList. Recuerde que el rango de índices para un ArrayList de tamaño listSize va
	de 0 a listSize–1. Por tanto, el número aleatorio calculado nos da un índice perfecto para
	acceder aleatoriamente a uno de los elementos de la lista completa.
	\\
	La última línea del método:\\
	-Extrae la respuesta situada en la posición index utilizando el método get.\\
	-Devuelve la cadena seleccionada como resultado del método, con la instrucción return.\\
	Si no tiene cuidado, su código podría generar un número aleatorio que quede fuera del rango de
	índices válidos del objeto ArrayList. Cuando luego intente utilizarlo como índice para acceder a
	un elemento de la lista, obtendrá un error IndexOutOfBoundsException.\\ 
	\\
	\textbf{Lactura de la documentación de las clases parametrizadas}\\
	Hasta ahora, le hemos pedido que examine la documentación de la clase String del paquete
	java.lang y de la clase Random del paquete java.util. Puede que haya observado al hacer
	esto que algunos nombres de clases en la lista contenida en la documentación tienen un aspecto
	ligeramente distinto, como por ejemplo ArrayList $<E>$ o HashMap$<K,V>$. Es decir, el nombre
	de la clase va seguido por una cierta información adicional que aparece entre corchetes angulares.
	Las clases de este estilo se denominan clases parametrizadas o clases genéricas.\\
	La información encerrada en los corchetes angulares nos dice que al utilizar estas clases debemos suministrar uno
	o más nombres de tipo entre corchetes angulares para completar la definición.\\
	Ya hemos visto aplicada
	esta idea en el Capítulo 4, donde hemos utilizado ArrayList parametrizándola con nombres
	de tipo como String. También pueden parametrizarse con cualquier otro tipo:\\
	private ArrayList$<String>$ notes;	
	 \\
	 private ArrayList$<Student>$ students;
	\\
	Dado que podemos parametrizar un ArrayList con cualquier otro tipo de clase que elijamos, este
	hecho se refleja en la documentación de la API. Por tanto, si examinamos la lista de métodos de
	ArrayList$<E>$, podremos ver métodos como:\\
	boolean add(E o)\\
	E get(int index)\\
	Esto nos dice que el tipo de objetos que podemos añadir a un ArrayList (con add) depende
	del tipo utilizado para parametrizarla y que el tipo de los objetos devueltos por su método get
	depende de la misma manera de ese tipo empleado en la parametrización. De hecho, si creamos un
	objeto ArrayList$<String>$, lo que la documentación nos dice es que el objeto tiene los siguientes
	dos métodos:\\
	boolean add(String o)\\
	String get(int index)\\
	mientras que si creamos un objeto ArrayLis$t<Student>$, entonces tendrá los otros dos métodos:\\
	boolean add(Student o)\\
	Student get(int index)\\
	\\
\subsection{Paquetes de imporacion}	

	import java.util.ArrayList;\\
	import java.util.Random;\\
	Las clases Java que están almacenadas en la librería de clases no están disponibles automáticamente
	para ser utilizadas. En lugar de ello,
	debemos indicar en nuestro código fuente que nos gustaría utilizar una clase de la librería. Esto
	se denomina importar la clase y se hace mediante la instrucción import. La instrucción import
	tiene el formato:\\
	import nombre-clase-cualificado;\\
	Dado que la librería Java contiene varios miles de clases, hace falta una cierta estructura en la organización
	de la librería para facilitar el manejo de ese gran número de clases. Java utiliza paquetes
	para clasificar las clases de librería en grupos de clases relacionadas. Los paquetes están anidados
	(es decir, los paquetes pueden contener otros paquetes).
	\\
	\\
	Las clases ArrayList y Random se encuentran ambas en el paquete java.util. Esta información
	puede encontrarse en la documentación de la clase. El nombre completo o nombre cualificado
	de una clase es el nombre de su paquete, seguido por un punto y por el nombre de la clase.
	Por tanto, los nombres cualificados de las dos clases que hemos usado aquí son: java.util.
	ArrayList y java.util.Random.\\
	Java también nos permite importar paquetes completos con instrucciones de la forma\\
	import nombre-paquete.*;\\
	Por tanto, la siguiente instrucción importaría todos los nombres de clase del paquete java.util:\\
	import java.util.*;\\
	Enumerar por separado todas las clases utilizadas, como en nuestra primera versión, requiere algo
	más de trabajo de escritura, pero resulta adecuado desde el punto de vista de la documentación.
	Indica claramente qué clases están siendo utilizadas realmente por nuestra clase. Por tanto, en este
	libro tenderemos a utilizar el estilo del primer ejemplo, enumerando por separado todas las clases
	importadas.\\
	Existe una excepción a estas reglas: algunas clases se emplean tan frecuentemente que casi todas
	las demás clases tendrán que importarlas. Estas clases se han incluido en el paquete java.lang,
	y este paquete se importa de manera automática en todas las clases. Por tanto, no necesitamos
	escribir instrucciones de importación para las clases contenidas en java.lang. La clase String
	es un ejemplo de ese tipo de clases.
	\subsection{Utilización de mapas para asociaciones}
	HashMap es una especialización de Map, que también está
	documentada. Se encontrará con que tiene que leer la documentación de ambas clases para comprender
	lo que es un HashMap y cómo funciona.\\
	Un mapa es una colección de parejas clave/valor de objetos. Como con ArrayList, un mapa
	puede almacenar un tipo flexible de entradas. Una diferencia entre ArrayList y Map es que con
	Map cada entrada no es un objeto, sino una pareja de objetos. Esta pareja está formada por un
	objeto clave y un objeto valor. En lugar de buscar entradas en esta colección utilizando un índice entero (como hicimos con
	ArrayList), empleamos el objeto clave para buscar el objeto valor.\\
	Un ejemplo de mapa en la vida cotidiana sería una guía telefónica. La guía telefónica contiene
	entradas y cada entrada es una pareja: un nombre y un número de teléfono. Utilizamos la guía
	telefónica buscando un nombre y leyendo el número de teléfono asociado. No empleamos ningún
	índice (la posición de la entrada en la guía) para averiguar el número de teléfono.
	Un mapa se puede organizar de tal manera que sea fácil buscar el valor correspondiente a una clave.
	En el caso de la guía telefónica, esto se hace mediante la ordenación alfabética. Si se almacenan
	las entradas en orden alfabético de sus claves, resulta sencillo localizar la clave y consultar el valor
	asociado. La búsqueda inversa (localizar la clave correspondiente a un valor, es decir, encontrar el
	nombre para un número de teléfono dado) no es tan simple con ayuda de un mapa. Al igual que
	con una guía telefónica, es posible realizar una búsqueda inversa en un mapa, pero se necesita un
	tiempo relativamente largo. Por tanto, los mapas son ideales para búsquedas en una sola dirección,
	en las que conocemos la clave de búsqueda y necesitamos saber el valor asociado con esa clave.\\
	\\
	HashMap es una implementación específica de Map. Los métodos más importantes de la clase
	HashMap son put y get.
	El método put inserta una entrada en el mapa, mientras que get extrae el valor correspondiente
	a una clave especificada. El siguiente fragmento de código crea un HashMap y inserta en él
	tres entradas. Cada entrada es una pareja clave/valor compuesta por un nombre y un número de
	teléfono.\\
	HashMap$<String, String>$ contacts = new HashMap$<>$();\\
	contacts.put("Charles Nguyen", "(531) 9392 4587");\\
	contacts.put("Lisa Jones", "(402) 4536 4674");\\
	contacts.put("William H. Smith", "(998) 5488 0123");\\
	Como hemos visto con ArrayList, al declarar una variable HashMap y al crear un objeto Hash-
	Map, tenemos que indicar qué tipo de objetos se almacenarán en el mapa y, adicionalmente, qué
	tipo de objetos se emplearán como clave. Para la guía de teléfonos utilizaríamos cadenas de caracteres
	tanto para las claves como para los valores, pero en otros casos ambos tipos serán diferentes.\\
	cuando se crean objetos de clases genéricas y se les asigna
	una variable, es preciso especificar los tipos genéricos en este caso $<String, String>$) solo una
	vez en el lado izquierdo de la asignación, y puede utilizarse el operador diamante en la construcción
	de objeto de la derecha; los tipos genéricos utilizados para la construcción de objeto se copian
	entonces de la declaración de variable.\\
	El siguiente código encuentra el número de teléfono de Lisa Jones y lo imprime.\\
	String number = phoneBook.get("Lisa Jones");\\
	System.out.println(number);\\
	Observe que pasamos la clave (el nombre “Lisa Jones”) al método get para recibir el valor (el
	número de teléfono).
	\subsection{Utilización de conjuntos}
	La librería estándar Java incluye diferentes variantes de conjuntos implementados en clases distintas.
	La clase que vamos a utilizar aquí se denomina HashSet.\\
	Los dos tipos de funcionalidad que necesitamos son la capacidad de introducir elementos en el
	conjunto y de extraer esos elementos posteriormente. Afortunadamente, estas tareas apenas contienen
	nada que nos resulte nuevo.\\
	import java.util.HashSet;\\
	HashSet$<String>$ mySet = new HashSet$<>$();
	\\
	mySet.add("one");\\
	mySet.add("two");\\
	mySet.add("three");
	\\
	Veamos cómo se iteraría a través de los elementos:\\
	for(String item : mySet)\{ \\
	\textit{hacer algo con ese elemento}
	\}\\
	En resumen, la utilización de colecciones en Java es bastante similar para los distintos tipos de
	colección. Una vez que se comprende cómo utilizar una de ellas, se pueden usar todas. Las diferencias
	radican, realmente, en el comportamiento de cada colección. Por ejemplo, una lista mantendrá
	todos los elementos que se introduzcan en el orden deseado, proporcionará acceso a esos
	elementos mediante un índice y puede contener el mismo elemento varias veces. Un conjunto,
	por el contrario, no mantiene ningún orden específico (los elementos pueden ser devueltos en un
	bucle for-each en un orden distinto de aquel en el que fueron introducidos) y garantiza que cada
	elemento se introduzca en el conjunto como máximo una vez. Introducir un elemento una segunda
	vez simplemente no tiene ningún efecto.
	\subsection{División de cadenas de caracteres}
	el método split, que es un método estándar
	de la clase String.
	El método split puede dividir una cadena en una serie de subcadenas separadas y devolverlas en
	una matriz de cadenas. (Se hablará más en detalle de matrices en el capítulo siguiente). El parámetro
	del método split define cuál es el tipo de caracteres según los cuales hay que dividir la cadena
	original. Lo que hemos hecho es definir que queremos cortar nuestra cadena por cada carácter de
	espaciado:
	\\
	String inputline = reader.nextLine().trim().toLowerCase();\\
	 String wordArray = inputline.split(" ");
	 \subsection{Autoboxing y clases envolventes}
	 con una parametrización adecuada, las clases de colección pueden almacenar
	 objetos de cualquier tipo. Sin embargo, persiste un problema: Java maneja algunos tipos que no
	 son tipos de objeto.\\
	 Como sabemos, los tipos sencillos (como int, boolean y char) son diferentes de los tipos de
	 objeto. Sus valores no son instancias de clases, y normalmente no sería posible añadirlos en una
	 colección.\\
	 Esto supone un problema. Se dan situaciones en las que, por ejemplo, podríamos querer crear una
	 lista de valores int o un conjunto de valores char.
	 La solución Java a este problema proviene de las clases envolventes (wrappers). Cada tipo primitivo
	 de Java tiene una clase envolvente correspondiente que representa el mismo tipo, pero que es
	 un tipo de objeto real. Por ejemplo, la clase envolvente para int recibe el nombre de Integer. Las instrucciones siguientes envuelven explícitamente el valor de la variable primitiva int ix en
	 un objeto Integer:\\
	 Integer iwrap = new Integer(ix);\\
	 \\
	 Así, por ejemplo, iwrap podría almacenarse fácilmente en una colección ArrayList$<Integer>$.
	 Sin embargo, el almacenamiento de valores primitivos en una colección de objetos se facilita
	 aún más mediante una característica del compilador denominada autoboxing.
	 \\
	 Siempre que se utilice un valor de un tipo de primitiva en un contexto que necesite una
	 clase envolvente, el compilador envuelve automáticamente el valor de tipo primitivo como un
	 objeto envolvente adecuado. Esto significa que los valores de tipo primitivo pueden añadirse
	 directamente a una colección:\\
	 private ArrayList$<Integer>$ markList;\\
	 public void storeMarkInList(int mark)
	 \{\\
	 	markList.add(mark);\\
	 \}\\
	 La operación inversa (unboxing) se realiza también de forma automática, con lo cual la recuperación
	 desde una colección tendría un aspecto como el siguiente:\\
	 int firstMark = markList.remove(0);\\
	 El autoboxing se aplica también siempre que se transfiera un valor de tipo primitivo como un
	 parámetro a un método que expera un tipo envolvente, y cuando se almacena un valor de tipo
	 primitivo en una variable de tipo envolvente. De forma semejante, el unboxing se aplica cuando
	 se pasa un valor de tipo envolvente como parámetro a un método que espera un valor de tipo
	 primitivo, y cuando se almacena en una variable de tipo primitivo. Conviene señalar que el resultado
	 es casi equivalente a almacenar los tipos primitivos en colecciones. Sin embargo, el tipo
	 de la colección aún debe declararse como un tipo envolvente (p. ej., ArrayList$<Integer>$,
	 no ArrayList$<int>$).
	 \subsection{Escritura de la documentación de las clases}
	 Cuando trabaje con sus propios proyectos, es importante que escriba la documentación de sus
	 clases a medida que desarrolle el código fuente. Los sistemas Java incluyen una herramientas denominada javadoc que puede utilizarse para
	 generar la descripción de esas interfaces a partir del código fuente. La documentación de la librería
	 estándar que hemos utilizado, por ejemplo, fue creada a partir del código fuente de las clases
	 mediante javadoc.\\
	 El entorno BlueJ utiliza javadoc para permitirnos crear documentación para nuestras clases de
	 dos formas distintas para:
	 -Ver la documentación para una única clase pasando el selector emergente situado en la parte
	 superior derecha de la ventana del editor de Source Code a Documentation, o seleccionando
	 Toggle Documentation View (Cambiar a vista de documentación) en el menú Tools (Herramientas)
	 del editor.\\
	 -Usar la función Project Documentation disponible en el menú Tools de la ventana principal para
	 generar la documentación correspondiente a todas las clases del proyecto.\\
	 \\
	 \textbf{Elementos de la documentación de una clase}
	 \\
	 La documentación de una clase debe incluir al menos:\\
	 -El nombre de la clase.\\
	 -Un comentario que describa el propósito global y las características de la clase.\\
	-Un número de versión.\\
	-El nombre del autor (o autores).\\
	-La documentación para cada constructor y cada método.\\
	La documentación de cada constructor y cada método debe incluir:\\
	-El nombre del método.
	-El tipo de retorno.\\
	- Los nombres y tipos de los parámetros
\\	
-Una descripción del propósito y función del método.\\
- Una descripción de cada parámetro.\\
-Una descripción del valor devuelto.\\
Además, cada proyecto completo ha de tener un comentario global del proyecto, que a menudo
estará contenido en un archivo “ReadMe” (Léame). En BlueJ, se puede acceder a este comentario
a través de la nota de texto mostrada en la esquina superior izquierda del diagrama de clases.\\
El símbolo de inicio del comentario tiene que tener dos asteriscos para que sea reconocido como
un comentario javadoc. Dicho comentario, si precede inmediatamente a la declaración de la
clase, se lee como un comentario de la clase. Si el comentario está justo encima de la signatura de
un método se considera un comentario del método.\\
Los detalles exactos de cómo se produce y formatea la documentación difieren en los distintos
lenguajes y entornos de programación. Sin embargo, el contenido debe ser más o menos siempre
el mismo.\\
En Java, con javadoc hay disponibles varios símbolos clave especiales para dar formato a la
documentación. Estos símbolos clave comienzan con el símbolo @ e incluyen:
@version
@author
@param
@return
\subsection{Public y private}
Los modificadores de acceso son las palabras clave public o private situadas al principio de las
declaraciones de campo y de las signaturas de método. Los campos, los constructores y los métodos pueden ser públicos o privados, aunque hasta ahora
hemos visto campos privados y constructores y métodos públicos.\\
Los modificadores de acceso definen la visibilidad de un campo, un constructor o un método. Por
ejemplo, si un método es público, se puede invocar desde la misma clase o desde cualquier otra.
Por el contrario, los métodos privados solo pueden invocarse desde la misma clase en la que están
declarados. No son visibles para otras clases.\\
Recuerde: la interfaz de una clase es el conjunto de detalles que necesita ver cualquier otro programador
que la use. Proporciona información acerca de cómo utilizar la clase. La interfaz incluye las
signaturas de los constructores y los métodos, además de una serie de comentarios. También se le
denomina parte pública de una clase. Su propósito es definir lo que la clase hace.
\\
La implementación es la sección de una clase que define precisamente cómo funciona la clase.
Los cuerpos de los métodos, que contienen las instrucciones Java y la mayoría de los campos son
parte de la implementación. La implementación también se denomina parte privada de una clase.
El usuario de una clase no necesita conocer su implementación. De hecho, hay buenas razones
por las que a un usuario debería impedírsele conocer la implementación (o al menos hacer uso de
dicho conocimiento). Este principio se conoce como ocultamiento de la información.\\
La palabra clave public declara un elemento de una clase (un campo o un método) como parte
de la interfaz (es decir, que es públicamente visible); la palabra clave private declara que es
parte de la implementación (es decir, que está oculto frente a posibles accesos externos).\\
Muy a menudo, el programador
de mantenimiento debe modificar o ampliar la implementación de una clase, con el fin de
realizar mejoras o de corregir errores. Idealmente, cambiar la implementación de una clase no
debería hacer necesario que se modifiquen otras clases. Esta cuestión se conoce también con el
nombre de acoplamiento. Si los cambios en una parte de un programa no hacen necesario realizar
cambios en otra parte de un programa, decimos que existe un acoplamiento bajo o débil.
El acoplamiento débil es positivo porque facilita notablemente el trabajo de un programador de
mantenimiento. En vez de tener que entender y modificar muchas clases, tal vez solo necesite
entender y modificar una única clase. Por ejemplo, si un programador de sistemas Java aporta una
mejora en la implementación de la clase ArrayList, confiamos en que no nos obligue a modificar
aquellas partes de nuestro código donde se use dicha clase. En principio, debería ser así, porque
no hemos hecho ninguna referencia a la implementación de ArrayList dentro de nuestro propio
código.\\
Por tanto, para ser más precisos, la regla de que a un usuario no debería permitírsele conocer
las interioridades de una clase no se refiere al programador de otra clase, sino a la propia clase.
Normalmente, no es ningún problema que un programador conozca los detalles de implementación,
pero las clases que ese programador desarrolle no deberían “conocer” (no deberían ser
dependientes de) los detalles internos de otra clase. El programador de ambas clases puede
ser incluso la misma persona, pero las clases deben estar débilmente acopladas.\\
\\
Otra buena razón para tener un método privado es cuando hace falta una tarea (como subtarea) en
varios de los métodos de una clase. En lugar de escribir el código múltiples veces, podemos escribirlo
una sola vez como un único método privado y luego invocarlo desde varios lugares distintos.\\
En Java, los campos también pueden declararse como privados o públicos. Hasta ahora, no hemos
visto ejemplos de campos públicos y hay una buena razón para ello. Declarar campos como públicos
viola el principio de ocultamiento de la información. Hace que una clase que dependa de dicha
información sea vulnerable a los fallos de operación, en caso de que la implementación cambie.
Aun cuando el lenguaje Java nos permite declarar campos públicos, consideramos que este es un
mal estilo de programación, así que no haremos uso de dicha opción.\\
Una razón adicional para mantener privados los campos es que otorga a los objetos un mayor
grado de control sobre su propio estado. Si canalizamos el acceso a un campo privado a través de
métodos selectores y mutadores, los objetos tendrán la posibilidad de garantizar que el campo no
se configure nunca con un valor que sea incoherente con su estado global.
\subsection{Palabra clabe static}
La palabra clave static es la sintaxis de Java para definir variables de clase. Las variables de
clase son campos que se almacenan en la propia clase, no en un objeto. En esto se diferencian fundamentalmente
de las variables de instancia\\
Como resultado,
habrá siempre una única copia de esta variable, independientemente del número de instancias que
creemos.
El código fuente de la clase puede acceder (leer y configurar) este tipo de variable exactamente
igual que en una variable de instancia. A la variable de clase se puede acceder desde cualquiera
de las instancias de la clase. Como resultado, todos los objetos de esa clase compartirán dicha
variable.\\
Las variables de clase se emplean frecuentemente cuando tenemos un valor que debe ser siempre
el mismo para todas las instancias de una clase. En lugar de almacenar una copia del mismo valor
en cada objeto, lo que sería un desperdicio de espacio y podría ser difícil de coordinar, se utiliza un
único valor compartido por todas las instancias.
Java también soporta los métodos de clase (también conocidos como métodos estáticos), que son
métodos que pertenecen a una clase.
\\
\\
\textbf{Constantes}\\
Un uso frecuente de la palabra clave static tiene lugar en la definición de constantes. Las constantes
son similares a las variables, pero no pueden cambiar de valor durante la ejecución de una
aplicación. En Java, las constantes se definen mediante la palabra clave final:\\
private final int SIZE = 10;\\
Aquí, hemos definido una constante denominada SIZE con el valor 10. Observe que las declaraciones
de constantes tienen un aspecto similar a las declaraciones de campos, con dos diferencias:\\
-Incluyen la palabra clave final antes del nombre del tipo.\\
-Han de inicializarse con un valor en el momento de la declaración.\\
Si se pretende que un valor no cambie nunca, es una buena idea declararlo como final. Esto
garantiza que no pueda ser modificado posteriormente de manera accidental. Cualquier intento
de modificar un campo constante provocará un mensaje de error en tiempo de compilación. Las
constantes se suelen escribir en mayúsculas\\
En la práctica es frecuente que las constantes se apliquen a todas las instancias de una clase. En
este tipo de situación, lo que hacemos es declarar constantes de clase. Las constantes de clase son
campos de clase constantes. Se declaran con una combinación de las palabras clave static y
final.\\
private static final int SIZE = 10;\\
\\
\textbf{Métodos de clase o estáticos}\\
Hasta ahora, todos los métodos que hemos visto han sido de tipo instancia: se invocan en una
instancia de una clase. La distinción entre los métodos de clase y los de instancia reside en que los
primeros pueden invocarse sin ninguna instancia: basta con tener la clase.\\
Para definir un método de clase se añade la palabra clave static delante del nombre de tipo en la
cabecera del método:\\
public static int getNumberOfDaysThisMonth()
\{
	...
\}\\
continuación puede llamarse al método especificando el nombre de la clase en la que se ha definido,
delante del punto en la notación habitual. Si, por ejemplo, el método anterior se define en una
clase denominada Calendar, se invoca mediante el código siguiente:\\
int days = Calendar.getNumberOfDaysThisMonth();\\
Como puede verse, el nombre de la clase se utiliza delante del punto; no se ha creado ningún
objeto.\\
Dado que los métodos de clase están asociados con una clase y no con una instancia, presentan
dos limitaciones importantes. La primera es que un método de clase tal vez no pueda acceder a
todos los campos de instancia definidos en la clase. Este resultado es lógico, ya que los campos de
instancia se asocian con objetos individuales. Así pues, los métodos de clase tienen restricciones
para acceder a las variables de clase desde su clase. La segunda limitación se parece a la primera:
un método de clase no puede invocar a un método de instancia desde la clase. Tan solo puede llamar
a otros métodos de clase definidos en su clase.
\subsection{El método principal}
Si queremos iniciar una aplicación Java sin BlueJ, debemos usar un método de clase. En BlueJ,
normalmente creamos un objeto e invocamos a uno de sus métodos, pero sin BlueJ, la aplicación
empieza sin que exista ningún objeto. Las clases son las únicas entidades con las que contamos
inicialmente, por lo cual el primer método al que debemos invocar es un método de clase.
La definición en Java para iniciar aplicaciones es bastante sencilla: el usuario especifica la clase
que debe iniciarse, y el sistema Java invocará un método denominado main en esta clase. Este
método debe tener una signatura muy determinada:\\
public static void main(String[] args)\\
Si en esa clase no existiera tal método se comunicaría un error.
\section{Colecciones de tamaño fijo}
El presente capítulo trata sobre colecciones que no son flexibles, sino que poseen una capacidad
fija; en el punto en que se crea el objeto de colección, tenemos que especificar el número máximo
de elementos que puede almacenar, que no podrá modificarse. A primera vista podría parecer una
restricción innecesaria, pero en algunas aplicaciones conocemos con antelación el número exacto
de elementos que deseamos almacenar en una colección, y ese número suele mantenerse fijo en
todo el tiempo de vida de la colección. En tales circunstancias, tenemos la opción de elegir utilizar
un objeto de colección de tamaño fijo especializado para almacenar los elementos.\\
\\
\textbf{Matrices:} Una colección de tamaño fijo se denomina matriz. Aunque el tamaño fijo de las matrices puede ser
una desventaja significativa en muchas situaciones, tienen a cambio dos ventajas frente a las clases
de colección de tamaño flexible:\\
-El acceso a los elementos almacenados en una matriz suele ser más eficiente que el acceso a los
elementos de una colección de tamaño flexible comparable.\\
-Las matrices pueden almacenar tanto objetos como valores de tipo primitivo. Las colecciones de
tamaño flexible solo pueden almacenar objetos\\
Otra característica distintiva de las matrices es que tiene un soporte sintáctico especial en Java; se
puede acceder a ellas con ayuda de una sintaxis personalizada que difiere de las llamadas tradicionales
a métodos. La razón es principalmente histórica: las matrices son la estructura de colección
más antigua utilizada en los lenguajes de programación, y la sintaxis para tratar con matrices se ha
ido desarrollando a lo largo de muchas décadas. Java utiliza la misma sintaxis establecida en otros
lenguajes de programación para hacer más sencillas las cosas para aquellos programadores que ya
estén empleando matrices, aun cuando este tratamiento no sea coherente con el resto de la sintaxis
del lenguaje.\\
La característica distintiva de la declaración de una variable de tipo matriz es una pareja de corchetes
que forman parte del nombre del tipo: int[]. Esto indica que la variable hourCounts es
del tipo matriz de enteros. Decimos en este caso que int es el tipo base de esta matriz concreta,
lo que significa que el objeto matriz almacenará valores de tipo int\\
La forma general de construcción de un objeto matriz es:
new tipo[expresión-entera]
La elección de tipo especifica el tipo de elemento que se almacenará en la matriz. La expresiónentera
especifica el tamaño de la matriz; es decir, el número fijo de elementos que puede almacenarse
en ella.\\
Cuando se asigna un objeto matriz a una variable matriz, el tipo del objeto matriz debe corresponderse
con el tipo declarado para la variable\\
A los elementos individuales de una matriz se accede indexando la matriz. Un índice es una expresión
entera escrita entre corchetes y que se coloca después del nombre de una variable de matriz. Los valores válidos para una expresión de índice dependen de la longitud de la matriz con la que
se esté trabajando. Al igual que sucede con otras colecciones, los índices de una matriz siempre
comienzan en cero y van hasta una unidad menos que la longitud de la matriz.\\
La utilización de un índice de matriz en el lado izquierdo de una instrucción de asignación es el
equivalente, dentro de las matrices, a un método mutador (o método set), porque se modificará
el contenido de la matriz. La utilización de un índice de matriz en cualquier otro lugar es el equivalente
de un método selector (o método get).
\subsection{El blucle for}
Java define dos variantes de bucles for, indicándose ambas mediante la palabra clave for en el
código fuente.
la primera de esas variantes, el bucle for-each,
que es un método conveniente para iterar a través de una colección de tamaño flexible. La segunda
variante, el bucle for, es una estructura de control iterativo alternativa2, que es particularmente
apropiada cuando:\\
-queremos ejecutar un cierto conjunto de instrucciones un número fijo de veces,\\
-necesitamos una variable dentro del bucle cuyo valor cambie en una cantidad fija, incrementándose
normalmente en 1, en cada iteración.\\
El bucle for está bien adaptado a aquellas situaciones en las que se necesita una iteración definida.
Por ejemplo, es común emplear un bucle for cuando queremos hacer algo con todos los elementos
de una matriz, como imprimir el contenido de cada elemento. Esto encaja con el criterio, ya que el
número fijo de veces se corresponde con la longitud de la matriz y nos hace falta una variable para
proporcionar un índice incremental para la matriz.\\
Un bucle for tiene la siguiente forma general:\\
for(inicialización; condición; acción post-cuerpo) \{ \\
	instrucciones que hay que repetir\\
\}\\
Cuando comparamos este bucle for con el bucle for-each, observamos que la diferencia sintáctica
se encuentra en la sección situada entre los paréntesis, en la cabecera del bucle. En este bucle for,
los paréntesis contienen tres secciones independientes, separadas por caracteres de punto y coma.\\
Aun cuando el bucle for se emplea a menudo para iteración definida, el hecho de que esté controlado
por una expresión booleana de carácter general indica que se aproxima más al bucle while que
al bucle for-each.\\
En general, cuando queremos acceder a todos los elementos de una matriz,
la cabecera del bucle for tendrá el siguiente formato general:\\
for(int index = 0; index $<$ array.length; index++)\\
\\
Existe un uso especial del bucle for con un Iterator cuando
deseamos hacer algo así. Suponga que deseáramos eliminar de nuestro organizador de música
todas las canciones de un artista concreto.podemos utilizar un bucle for de la
manera siguiente:\\
for(Iterator$<Track>$ it = tracks.iterator(); it.hasNext(); ) \{\\
Track t = it.next();\\
if(track.getArtist().equals(artist)) \{\\
t.remove();\\
\}\\
\}\\
El aspecto importante aquí es que no hay ninguna acción post-cuerpo del bucle; nos hemos limitado
a dejarla en blanco. Esto es perfectamente legal, pero seguimos teniendo que incluir el punto
y coma después de la condición del bucle. Utilizando un bucle for en lugar de un bucle while,
queda algo más claro que pretendemos examinar todos los elementos de la lista.
\subsection{
El operador condicional}
if(condición) \{\\
	hacer algo;\\
\}
else \{\\
	hacer algo similar;\\
\}\\
Este operador se utiliza para seleccionar uno de los dos valores alternativos,
basándose en la evaluación de una expresión booleana y la forma general es:\\
condición ? valor1 : valor2\\
Si la condición es true, entonces el valor de la expresión completa será valor1, y en caso contrario
será valor2. 
\\
Códigos de Wolfram. En 1983, Stephen Wolfram publicó un estudio con los 256 autómatas
celulares elementales posibles. Propuso un sistema numérico para definir la conducta de cada tipo
de autómata, y el código asignado a cada uno recibe el nombre de código de Wolfram. A partir del
código numérico, es muy sencillo deducir la regla aplicable para los cambios de estado, dados los
valores de una célula y de sus dos vecinas, dado que el mismo código codifica las reglas. Véase el
Ejercicio 7.36, mostrado más adelante, sobre el funcionamiento en la práctica, o busque en la web los
términos “elementary cellular automata” (autómatas celulares elementales).\\
\\
La sintaxis para declarar una variable de matriz de más de una dimensión es una extensión del caso
unidimensional: un par de corchetes vacíos por cada dimensión:\\
Cell[][] cells;\\
De forma semejante, la creación del objeto matriz mediante new especifica la longitud de cada
dimensión:\\
cells = new Cell[numRows][numCols];
\section{Diseño de clases}
Los malos diseños tienen más que ver con las decisiones que tomamos a la hora de resolver un
problema concreto. No podemos utilizar como excusa para hacer un mal diseño el argumento de
que no había otra manera de resolver el problema.
\subsection{Acoplamiento y cohesión}
Hay dos términos que son fundamentales a la hora de hablar de
la calidad de un diseño de clase: el acoplamiento y la cohesión.\\
El término acoplamiento hace referencia al grado de interconexión de las clases. lo que buscamos es diseñar nuestra aplicación como un conjunto de clases
en cooperación, que se comunican a través de interfaces bien definidas. El grado de acoplamiento
indica lo estrechamente conectadas que están las clases. Lo que buscamos es un grado bajo de
acoplamiento, o un acoplamiento débil.\\
\\
El grado de acoplamiento determina lo difícil que es realizar cambios en una aplicación. En una
estructura de clases estrechamente acoplada, un cambio en una clase puede hacer necesario introducir
cambios también en otras clases. Esto es precisamente lo que tratamos de evitar, porque el
efecto de realizar un pequeño cambio puede propagarse rápidamente en cascada a través de toda la
aplicación. Además, localizar todos los lugares en los que es necesario hacer cambios y llevar a
la práctica esos cambios puede resultar difícil y requerir mucho tiempo. En un sistema débilmente acoplado, por el contrario, podemos cambiar una clase sin efectuar ningún
cambio en las clases restantes, y la aplicación seguirá funcionando correctamente.\\
\\
El término cohesión se relaciona con el número y la diversidad de las tareas de las que es responsable
cada unidad de una aplicación. La cohesión es relevante tanto para unidades formadas por una
sola clase, como para métodos individuales\\
Idealmente, cada unidad de código debe ser responsable de una tarea coherente (es decir, una tarea
que pueda ser vista como una unidad lógica). Cada método debería implementar una operación
lógica, y cada clase debería representar un tipo de entidad. La principal razón que subyace al
principio de la cohesión es la reutilización: si un método o clase es responsable de una única cosa
bien definida, entonces es mucho más probable que pueda utilizarse de nuevo en un contexto distinto.
Una ventaja complementaria de adherirse a este principio es que, cuando haga falta realizar
modificaciones en algún aspecto de la aplicación, es probable que encontremos todas las piezas
relevantes dentro de una misma unidad.\\
\\
La duplicación de código es un indicador de un mal diseño. El problema con la duplicación de código es
que cualquier modificación en una versión debe ser realizado también en la otra, si queremos evitar
las incoherencias. Esto incrementa la cantidad de trabajo que el programador de mantenimiento
tiene que realizar e introduce el peligro de que aparezcan errores.\\
Normalmente, la duplicación de código es un síntoma de una mala cohesión.
\subsection{Encapsulación para reducir el acoplamiento}
La directriz referida a la encapsulación (ocultar la información de la implementación a ojos de
otras clases) sugiere que solo debe hacerse visible para el exterior la información acerca de lo que
hace una clase, no la información acerca de cómo lo hace. Esto tiene una gran ventaja: si ninguna
otra clase sabe cómo está almacenada nuestra información, entonces podemos cambiar fácilmente
el modo en que está almacenada sin por ello hacer que otras clases dejen de funcionar.\\
Podemos obligar a esta separación entre lo que se hace y cómo se hace definiendo los campos
como privados y utilizando un método selector para acceder a ellos.
\subsection{Diseño dirigido por responsabilidad}
Hemos visto en la sección anterior que el hacer uso de una encapsulación adecuada reduce el
acoplamiento y puede disminuir significativamente la cantidad de trabajo necesario para llevar a
cabo cambios en una aplicación. Sin embargo, la encapsulación no es el único factor que influye
en el grado de acoplamiento. Otro aspecto es el conocido con el nombre de diseño dirigido por
responsabilidad.\\
El diseño dirigido por responsabilidad expresa la idea de que cada clase debe ser responsable de
gestionar sus propios datos.
A menudo, cuando necesitamos añadir alguna nueva funcionalidad a una aplicación, tenemos que preguntarnos a nosotros mismos en qué clase deberíamos añadir
un método para implementar esa nueva función. ¿Qué clase debería ser responsable de la tarea?
La respuesta es que la clase responsable de almacenar unos determinados datos debería ser también
responsable de manipularlos. Lo bien que se utilice el diseño dirigido por responsabilidad influye en el grado de acoplamiento y,
por tanto, influye de nuevo en la facilidad con la que se puede modificar o ampliar una aplicación.
\\
Otro aspecto de los principios de acoplamiento y responsabilidad es el de la localidad de los cambios.
Lo que queremos es crear un diseño de clases que facilite los cambios posteriores, haciendo
que los efectos de un cambio sean locales.
Idealmente, solo deberíamos tener que cambiar una única clase para realizar una modificación.
En ocasiones, será necesario modificar varias clases, pero entonces nuestro objetivo será que se
trate del menor número de clases posible. Además, los cambios necesarios en otras clases deberán
ser obvios, fáciles de detectar y sencillos de llevar a cabo.\\
En buena medida, podemos conseguir esto siguiendo unas buenas reglas de diseño, como por
ejemplo emplear un diseño dirigido por responsabilidad y tratar de conseguir un acoplamiento
débil y una alta cohesión. Además, sin embargo, deberíamos tener en mente las futuras modificaciones
y ampliaciones en el momento de crear nuestras aplicaciones. Es importante prever que
un cierto aspecto de nuestro programa puede cambiar de cara a hacer que dicho cambio sea lo más
fácil posible.
\\
\\

\textbf{Acomplamiento implícito:} El acoplamiento implícito es una situación en la que una clase depende de la información interna de
otra, pero dicha dependencia no es inmediatamente obvia. El acoplamiento fuerte en el caso de los
campos públicos no era bueno, pero al menos era obvio: si cambiamos los campos públicos en una
clase y nos olvidamos de la otra, la aplicación no podrá compilarse y el compilador nos indicará
que hay un problema. Sin embargo, en los casos de acoplamiento implícito, la omisión de un cambio
necesario puede no ser detectada.\\
Se trata de un problema fundamental, porque cada vez que añadimos un comando, tenemos que
cambiar el texto de ayuda y es bastante sencillo olvidarse de hacer este cambio. El programa se
compila y se ejecuta y todo parece ir correctamente. El programador de mantenimiento podría
creer que su tarea ha finalizado y lanzar comercialmente un programa que ahora contendrá un
error.
Este es un ejemplo de acoplamiento implícito. Cuando los comandos cambian, el texto de ayuda
debe modificarse (acoplamiento), pero no hay nada en el código fuente del programa que indique
claramente esta dependencia (y es por ello que es implícita).
Una clase bien diseñada evitaría esta forma de acoplamiento siguiendo la regla del diseño dirigido
por responsabilidad. \\
\\principio de cohesion: cada unidad de código debería ser
siempre responsable de una, y solo una, tarea. Cuando hablamos de la cohesión de métodos, queremos expresar que, idealmente, cada método
debe ser responsable de una, y solo una, tarea bien definida. El principio de cohesión se puede aplicar a clases y métodos: tanto unas como otros deben mostrar
un alto grado de cohesión. \\
Sin embargo, es mucho más fácil entender lo que hace un segmento de código y es también más
fácil realizar modificaciones en el mismo, si se utilizan métodos cortos y bien cohesionados. Con
la estructura de métodos elegida, todos los métodos son relativamente cortos y fáciles de entender,
y sus nombres indican sus propósito de forma bastante clara. Estas características representan una
ayuda muy valiosa para el programador de mantenimiento.\\
\\
\textbf{Cohesión de clase:} La regla de la cohesión de las clases afirma que cada clase debería representar una única entidad
bien definida dentro del dominio del problema.\\
Son varias las maneras en que una alta cohesión beneficia a un diseño. Las dos más importantes
son la legibilidad y la reutilización.
\subsection{Refactorización}
Al diseñar aplicaciones, debemos tratar de planificar por adelantado, anticipando los posibles cambios
que puedan realizarse en el futuro y creando clases y métodos altamente cohesionados y
débilmente acoplados que faciliten las modificaciones. Este es un objetivo muy loable, pero por
supuesto no siempre podemos anticipar todas las futuras adaptaciones y tampoco es factible prepararse
para todas las posibles ampliaciones que podamos imaginar.
Esta es la razón de que la refactorización sea importante.\\
\\
La refactorización es la actividad consistente en reestructurar las clases y método existentes, para
adaptarlos a los cambios en la funcionalidad y en los requisitos. A menudo, a lo largo de la vida de
una aplicación se suele ir añadiendo gradualmente funcionalidad. Un efecto común es que, como
consecuencia directa, la longitud de los métodos y de las clases crece lentamente.
Resulta tentador para un programador de mantenimiento añadir código adicional a las clases o
método existentes. Sin embargo, si seguimos haciendo esto durante un cierto tiempo, el grado de
cohesión se reducirá. Cuando se añade más y más código a un método o una clase, es probable
que llegue un momento en el que ese método o esa clase representen más de una tarea o entidad
claramente definida.
La refactorización consiste en repensar y rediseñar las estructuras de clases y métodos. El efecto
más común es que las clases se dividan en dos o que los métodos se dividan en dos o más métodos.
La refactorización puede incluir también la unión de varias clases o métodos en uno, aunque esto
suele ser bastante menos común que la división.\\
\\
Antes de proporcionar un ejemplo de refactorización, tenemos que reflexionar sobre el hecho de
que, al refactorizar un programa, estamos proponiendo normalmente realizar cambios potencialmente
grandes en algo que ya funciona. Cuando se modifica algo, hay una cierta probabilidad de
que se produzcan errores. Por tanto, es importante actuar con cautela y, antes de refactorizar, debemos
asegurarnos de que exista un conjunto de pruebas para la versión actual del programa. Si esas
pruebas no existen, entonces debemos decidir primero cómo se puede probar razonablemente la
funcionalidad del programa, y dejar constancia de esas pruebas (por ejemplo, anotándolas por
escrito) de modo que podamos repetir esas mismas pruebas posteriormente.\\
\\
Idealmente, la
refactorización debe realizarse en dos etapas:\\
La primera etapa consiste en refactorizar para mejorar la estructura interna del código, pero
sin realizar ningún cambio en la funcionalidad de la aplicación. En otras palabras, el programa
debería, al ejecutarse, comportarse de la misma forma exacta que antes. Una vez completada
esta etapa, habrá que repetir las pruebas previamente decididas para verificar que no hemos
introducido errores inadvertidamente.\\
 La segunda etapa puede acometerse una una vez que hayamos restablecido la funcionalidad base en
 la versión refactorizada. Entonces estaremos en disposición de mejorar el programa. Una vez
 que lo hayamos hecho, por supuesto, habrá que realizar pruebas con la nueva versión.\\
 Hacer varios cambios al mismo tiempo (refactorizar y añadir nuevas características) hace que sea
 más difícil localizar las fuentes de error en caso de que introduzcan errores.

\section{Objetos don un buen comportamiento}
\end{document}